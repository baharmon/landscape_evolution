
% --------------------------- EI_30 ---------------------------------------

%% single storm erosivity
%\noindent The erosivity of a single rainfall event $EI_{30}$
%divided into $m$ total periods
%can be calculated as
%\citep{Panagos2015,Panagos2017}:
%%
%\begin{equation}
%\label{eq:erosivity_index}
%EI_{30} = \Bigg(\sum_{r=1}^{m}~ e_r ~ v_r \Bigg) ~ I_{30}
%\end{equation}
%%
%{\small
%\noindent
%where: \\
%\hspace*{0.5em} $EI_{30}$ is the rainfall erosivity index of a single event [\unit{MJ~mm~ha}$^{-1}$~\unit{hr}$^{-1}$]\\
%\hspace*{0.5em} $e_r$ is the unit rainfall energy [\unit{MJ~ha}$^{-1}$~\unit{mm}$^{-1}$]\\
%\hspace*{0.5em} $v_r$ is the rainfall volume [\unit{mm}] during time period $r$\\
%\hspace*{0.5em} $I_{30}$ is the maximum rainfall intensity during a 30-min period of the rainfall event [\unit{mm}~\unit{hr}$^{-1}$].
%}
%%
%\noindent In this model, however, the erosivity factor
%is derived at each time step $r$ as a function of
%kinetic energy ($e_r$), rainfall volume ($v_r$), rainfall intensity ($i_r$). 
%First rain energy is derived from rainfall intensity \citep{Brown1987}: 

%\begin{equation}
%\label{eq:rain_energy}
%{e_r = 0.29 ~ (1.-0.72 ~ exp(-0.05 ~ i_r))}
%\end{equation}
%%
%{\small
%\noindent
%where: \\
%\noindent
%\hspace*{0.5em} $e_r$is unit rain energy [\unit{MJ~ha}$^{-1}$~\unit{mm}${^-1}$]\\
%\hspace*{0.5em} $i_r$ is rainfall intensity [\unit{mm~h}$^{-1}$].\\
%}

%% erosivity per time step
%\noindent
%Then the erosivity per time step %$EI_r$ 
%is calculated as the product of 
%unit rain energy, rainfall volume, rainfall intensity, and time:  
%%
%\begin{equation}
%\label{eq:erosivity_index}
%EI_{r} = e_r ~ v_r ~ i_{r}
%\end{equation}
%%
%{\small
%\noindent
%where: \\
%\hspace*{0.5em} $EI_r$ is the rainfall erosivity during time $r$ [\unit{MJ~mm~ha}$^{-1}$~\unit{hr}$^{-1}$].
%%\hspace*{0.5em} $e_r$ is the unit rainfall energy [\unit{MJ~ha}$^{-1}$~\unit{mm}$^{-1}$]\\
%%\hspace*{0.5em} $v_r$ is the rainfall volume [\unit{mm}] during time period $r$\\
%%\hspace*{0.5em} $i_r$ is the rainfall intensity during time period $r$ [\unit{mm}~\unit{hr}$^{-1}$].
%}