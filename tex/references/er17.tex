%%%%%%%%%% espcrc1.tex %%%%%%%%%%
%
% $Id: espcrc1.tex 1.2 2000/07/24 09:12:51 spepping Exp spepping $
%
\documentclass[fleqn,12pt,twoside]{article}
\usepackage{espcrc1}

% change this to the following line for use with LaTeX2.09
% \documentstyle[12pt,twoside,fleqn,espcrc1]{article}

% if you want to include PostScript figures
\usepackage{graphicx}
% if you have landscape tables
\usepackage[figuresright]{rotating}

% put your own definitions here:
%   \newcommand{\cZ}{\cal{Z}}
%   \newtheorem{def}{Definition}[section]
%   ...
\newcommand{\ttbs}{\char'134}
\newcommand{\AmS}{{\protect\the\textfont2
  A\kern-.1667em\lower.5ex\hbox{M}\kern-.125emS}}

% add words to TeX's hyphenation exception list
\hyphenation{author another created financial paper re-commend-ed Post-Script}

% declarations for front matter
\title{
Path sampling method for modeling overland water flow, sediment
transport and short term terrain evolution in Open Source GIS}

\author{Helena Mitasova\address{
Department of Marine, Earth and Atmospheric Sciences
North Carolina State University
1125 Jordan Hall, Raleigh, NC 27695, USA},
Chris Thaxton\address[DPNCSU]{Department of Physics, North Carolina State University,
Raleigh, NC 27695, USA},
Jaroslav Hofierka\address{Department of Geography and Geoecology, Presov University,
Presov, Slovakia},
%{Department of Physics, North Carolina State University, Raleigh, NC 27695-8208, USA},
Richard McLaughlin\address[DSS]{Department of Soil Science,
North Carolina State University, Raleigh, NC 27695, USA},
Amber Moore\addressmark[DSS],
Lubos Mitas \addressmark[DPNCSU]}
%{Department of Soil Science, North Carolina State University, admoore2@unity.ncsu.edu

\begin{document}

% typeset front matter
\maketitle

\begin{abstract}
A path sampling method is proposed 
for solving the continuity equations describing mass flows over complex landscape 
surfaces. The modeled quantities are
represented by an ensemble of sampling points which are evolved according to
the corresponding Green function. The method enables incorporation of
multi-scale/multi-process treatments. It has been used to develop simulation 
tools for overland shallow water flow and for sediment transport. Spatial
pattern of sediment flow and net erosion/deposition is modeled using the closure
relationship between sediment transport capacity and detachment developed for
the USDA Water Erosion Prediction Project. The tools were recently implemented
as modules in Open Source GRASS GIS. Their application is illustrated by
the study of impact of land use and topography change on overland flow and
sediment transport at North Carolina State University campus.
\end{abstract}

\section{INTRODUCTION}

The emergence of new mapping and automated monitoring technologies has created
opportunities to improve the predictions of anthropogenic impacts on landscapes
and to find sustainable solutions for development. Modeling of landscape processes 
plays an important role in this effort by allowing us to simulate the impact of proposed changes
before they are implemented, and by providing tools to explore
a wide range of alternatives. The new, high resolution data along with
the new type of distributed models have a potential to bring the simulations
to the level of realism and accuracy needed for decision making.

Traditional, spatially averaged models 
%(e.g. \cite{swat94})
have limited capabilities to identify locations of problems
(e.g. pollution sources) and the pattern of their propagation through landscapes.
Also the possibilities to explore various alternatives of land use and optimal
organization of landscape are restricted.
Models based on continuous fields provide such insight,
however, they are much more complicated in terms of their implementation
and data requirements. Coupling with a Geographic Information System (GIS)
makes applications of these models more efficient making them feasible 
for practical applications.
In early 90ies, Geographic Resources Analysis and Support System (GRASS),
 as a public domain GIS,  provided an environment
for pioneering work in integrating GIS and landscape process modeling
(e.g., \cite{answers91}, \cite{swat94}, \cite{saghafian96}, \cite{vieux96}).
While most of these models are now linked to proprietary GIS,
the release of GRASS5.0 within the open source computational infrastructure \cite{grassbook}
creates opportunities for development of new generation distributed models. 
Fully disclosed and available source code provides
specialized libraries which make software implementation of the model simpler
and more effective.

Current spatially distributed models, such as SIBERIA: \cite{siberia95};
CASC2d: \cite{casc2d95}, CHILD: \cite{child01},
SIMWE: \cite{mitwrr98},   have common theoretical foundation,
however, the implementations were developed for different type of applications.
In this paper, we focus on modeling approach that supports spatial analysis
of short term erosion/deposition patterns in landscapes 
with complex topography and land cover distribution, and provides information 
needed for better conservation and erosion prevention planning.

%We present a full GIS integration of a path sampling method
%for modeling fluxes represented by continuity equation with advection
%and diffusion term (e.g. overland water flow) and with added rate
%term describing the local rate of proliferation or decay (deposition)
%of modeled substance. The method incorporates both deterministic
%and stochastic influences.


\section{METHODS}

%Methods used for the first version of model were presented by Mitas and Mitasova
%\cite{mitwrr98}, \cite{mitlle02}, here we extend it to a dynamic terrain
%evolution model by combining finite difference in time with path sampling in space ??? .
To simulate the impact of a given configuration of topography, land cover,
soil properties and rainfall event on spatial pattern of erosion/deposition
and terrain evolution, we first solve the bivariate shallow water flow
equation to obtain the spatial distribution of water flow depth. 
The water depth field is then used as an 
input to sediment transport model to estimate the distribution 
of net erosion/deposition and change in the elevation surface. 
The following sections describe the equations 
and numerical solution in more detail.

\subsection{Shallow overland flow}

For a shallow water flow, spatial variation in velocity with respect to depth
can be neglected and the flow process can be approximated by
the bivariate form of the St Venant equation \cite{casc2d95}:  

\begin{equation}
\label{eq:water}
{\partial h({\bf r},t) \over \partial t} =
 i_e({\bf r},t) - \nabla \cdot {\bf q}({\bf r},t)
\end{equation}
while the momentum conservation in the diffusion wave approximation
has the form:

\begin{equation}
\label{eq:momentum}
{\bf s_f}({\bf r},t)={\bf s}({\bf r})-\nabla h({\bf r},t)
\end{equation}

\noindent
where 
${\bf r}=(x,y)$ [m] is the position, 
$t$ [s] is the time,
$h({\bf r},t)$ [m] is the depth of overland flow,
$i_e({\bf r},t)$ [m/s] is the rainfall excess = 
(rainfall $-$ infiltration $-$ vegetation intercept) [m/s],
${\bf q}({\bf r},t)$ [$\rm m^2/s$] is the water flow per unit width,
${\bf s}({\bf r})= - \nabla z({\bf r})$ is the negative elevation gradient,
$z({\bf r})$ [m] is the elevation,
and ${\bf s_f}({\bf r},t)$ is the negative gradient of
overland flow surface (friction slope).
For a shallow water flow, with the hydraulic radius approximated
by the normal flow depth $h({\bf r},t)$ \cite{moorefoster90}
the unit discharge is given by:

\begin{equation}
\label{eq:flowdepth}
{\bf q}({\bf r},t)={\bf v}({\bf r},t) h({\bf r},t),
%\alpha h^m {{\nabla z({\bf r})} \over {|\nabla z({\bf r})|}}
%\eqno(3)
\end{equation}

\noindent
where ${\bf v}({\bf r},t)$  [m/s] is the flow velocity.
The system of equations (\ref{eq:water}) - (\ref{eq:flowdepth}) is closed
using the Manning's relation between $h({\bf r},t)$ and
 ${\bf v}({\bf r},t)$:

\begin{equation}
\label{eq:man}
{\bf v}({\bf r},t)=
{C\over n({\bf r})} h({\bf r},t)^{2/3}
|{\bf s_f}({\bf r},t)|^{1/2} {\bf s_{f0}({\bf r},t)}
%\eqno(4)
\end{equation}

\noindent
where 
$n({\bf r}) $ is the dimensionless Manning's coefficient, 
$C=1$ [$m^{1/3}/s$] is the corresponding dimension constant \cite{ding84},
and ${\bf s_{f0}}({\bf r})={\bf s_f}({\bf r})/|{\bf s_f}({\bf r})|$
is the unit vector in the friction slope direction.
To account for spatially variable cover, necessary for
land use management, we consider $n({\bf r})$ and $i_e({\bf r},t)$
 as explicitly location dependent.

To model erosion/deposition patterns and short term evolution of topography,
we use the solution of continuity and momentum equations for 
a steady water flow that is close to kinematic wave approximation:
%In addition, the flow is considered to be close to the
%kinematic wave approximation
% for which ${\bf s_f}({\bf r},t) \approx {\bf s}({\bf r})$ and after
%using Equation \ref{eq:flowdepth}, the Equation \ref{eq:water} %is given by:

\begin{equation}
\label{eq:sswater}
\partial h({\bf r},t)/\partial t =0 
\quad \longrightarrow \quad
\nabla \cdot [ h({\bf r}){\bf v}({\bf r})] = i_e({\bf r})
%\eqno(5)
\end{equation}
In order to incorporate the diffusive wave effects at least in an approximate way,
we introduce a diffusion-like term
$ \propto \nabla^2 [h^{5/3}({\bf r})]$ into equation (\ref{eq:sswater}):

\begin{equation}
\label{eq:difwater}
-{\varepsilon({\bf r})\over 2 }\nabla^2 [h^{5/3}({\bf r})]
+\nabla \cdot [ h({\bf r}){\bf v}({\bf r})] = i_e({\bf r})
%\eqno (6)
\end{equation}

\noindent
 where $\varepsilon({\bf r})$ is a spatially variable diffusion coefficient.
%Such an incorporation of diffusion in the water flow simulation is not new
% and a similar term has been obtained in derivations of diffusion-advection
%equations for overland flow, e.g.,  by Dingman, (1984) and Lettenmeier and Wood, (1992).
The diffusion term, which depends on $h^{5/3}({\bf r})$ instead of
$h({\bf r})$, makes the equation (\ref{eq:difwater})  linear
in the function $h^{5/3}({\bf r})$ which enables it to be
solved using the path sampling method.

%write this in math expression
%v.to.rast derives raster map for streams with 1/0 and aspect/0
%dx and dy is derived for the streams (slope is 2):
%mapcalc> str.dx10=tan(2)*cos(hydrodir10)
%mapcalc> str.dy10=tan(2)*sin(hydrodir10)
%combined dx,dy map is created from dx,dy for elevations and streams:
%mapcalc> fstr10.55dxa=if(hydro10,str.dx10,fs10.55dx)
%mapcalc> fstr10.55dya=if(hydro10,str.dy10,fs10.55dy)

\subsection{Sediment transport, soil erosion/deposition and terrain evolution}

The general equation for change in topography due to erosion and deposition, adapted
from \cite{karambas02} to three dimensions, is:

\begin{equation}
\label{eq:elevchange}
{\partial z({\bf r},t) \over \partial t} = - {1 \over \varrho_b({\bf r})} \nabla\cdot
{\bf q'_s}({\bf r},t)
\end{equation}
where
\begin{equation}
\label{eq:elevchange2}
{\bf q'_s}({\bf r},t) = {\bf q_s}({\bf r},t) - \gamma({\bf r}) |{\bf q_s}({\bf r},t)| \nabla z
\end{equation}
Here, $z({\bf r},t)$ [m] is the elevation at location ${\bf r}$ and time $t$,
$\varrho_b({\bf r})$ is bulk density [$kg/m^{3}$] and $\gamma$ is a diffusion coefficient.
In it's basic form, equation (\ref{eq:elevchange}) is simply the conservation equation
for sediment mass commonly used in models that incorporate elevation change
(e.g. \cite{child01}, \cite{parker00}, \cite{paola95}, \cite{dietrich93}).
A gravitational diffusion term captures the influence of local terrain slope
on sediment transport (\cite{horikawa88}): the resulting {\it effective sediment load}
(\ref{eq:elevchange2}) is either increased or decreased, delaying deposition
(downslope) or encouraging it (upslope), respectively. The diffusion term
is the primary mechanism for preventing the terrain from growing indefinitely
and allowing the terrain to reach a state of dynamic equilibrium.

%Under the current implementation we simulate erosion for a given steady water depth -
%by couplig with water flow dynamics we can simulate evolution of erosion deposition
To estimate the sediment flow and net erosion/deposition
we use the sediment continuity equation,
which relates the change in sediment storage over time, and the change
in sediment flow rate along the hillslope to effective sources and sinks
\cite{haan94}, \cite{govin91}:

\begin{equation}
\label{eq:sediment}
{\partial [\rho_sc({\bf r},t)h({\bf r},t)] \over \partial t} +
\nabla\cdot {\bf q_s}({\bf r},t) = {\rm sources - sinks}=
D({\bf r},t)
%\eqno(7)
\end{equation}

\noindent
where
${\bf q_s}({\bf r},t)$ [kg/(ms)] is the sediment flow rate per unit width,
$c({\bf r},t)$ [particle/m$^3$] is sediment concentration,
$\rho_s$ [kg/particle] is mass per sediment particle,
$\rho_s c({\bf r},t)$ [kg/m$^3$] is sediment mass density,
and $D({\bf r},t)$ [kg/(m$^2$s)] is the net erosion or deposition rate.
The sediment flow rate ${\bf q_s}({\bf r},t) $ is a function of
water flow and sediment concentration:

\begin{equation}
\label{eq:sedflow}
  {\bf q_s}({\bf r},t) = \rho_sc({\bf r},t){\bf q}({\bf r},t)
%\eqno (8)
\end{equation}

\noindent
Again, we assume a steady state form of
%For shallow, gradually varied flow the storage term can be
%neglected leading to a steady state form of
the continuity equation:
\begin{equation}
\label{eq:ssediment}
{\partial [\rho_sc({\bf r},t)h({\bf r},t)] \over \partial t}=0
\quad
\longrightarrow \quad
\nabla\cdot {\bf q_s}({\bf r}) = D({\bf r}) .
%\eqno(9)
\end{equation}
\noindent
%where $D({\bf r})\; [kg/(m^2s)]$ is the net erosion/deposition rate.
The sources and sinks term is derived from the assumption that
the detachment and deposition rates are proportional to the difference between
the sediment transport capacity and the actual sediment flow rate \cite{fostermeyer72}:

\begin{equation}
D({\bf r})
=\sigma({\bf r}) \bigl[ T({\bf r})- |{\bf q_s}({\bf r})|\bigr]
%\eqno(10)
\end{equation}

\noindent
where $T({\bf r})$ [kg/(ms)] is the sediment transport capacity,
$\sigma({\bf r})$ [m$^{-1}$] is the first order reaction term
 dependent on soil and cover properties.
 The expression for $\sigma({\bf r})=D_c({\bf r})/T({\bf r})$
is obtained from the following relationship \cite{fostermeyer72}:

\begin{equation}
D({\bf r})/D_c({\bf r}) + |{\bf q_s}({\bf r})|/T({\bf r}) = 1
%\eqno(11)
\end{equation}

\noindent
% which states that the ratio of the erosion rate to the detachment
% capacity $D_c ({\bf r})$ in $kg/(m^2s)$ %(entrainment)
% plus the ratio of the sediment flow to the sediment transport capacity is
% a conserved quantity (unity). To keep the model simple, $\sigma({\bf r})$
%is applied for both net erosion and deposition.
%This equation (11) is based on the observed relationship between 
%soil detachment and transport described e.g., by Meyer and Wischmeier
%(1969). This concept is used in several erosion models including {\sl WEPP}
%(Haan et al., 1994; Flanagan and Nearing, 1995).
The qualitative arguments, experimental observations and values for
 $\sigma({\bf r})$   are discussed, for example, by Foster and Meyer\cite{fostermeyer72}.
%and Foster (1982). While it is possible to use other frameworks
%for estimation of  $\sigma({\bf r})$, we have chosen the Foster and
%Meyer concept because of its simplicity and widespread use.
The sediment transport capacity $T({\bf r})$ and detachment capacity $D_c({\bf r})$ 
%represent the maximum potential
%sediment flow rate and the maximum potential detachment rate, respectively,
%by overland flow
can be expressed by numerous simplified equations representing these rates under
different conditions. In the WEPP model \cite{wepp95}, they are expressed as functions
of a shear stress:

\begin{equation}
T({\bf r})=K_t({\bf r}) \bigl[\tau({\bf r})\bigr]^p
%=K_t({\bf r}) \bigl[\rho_w\, g h({\bf r}) \sin \beta ({\bf r}) \bigr]^p
%\eqno(12)
\end{equation}
\begin{equation}
D_c({\bf r})=K_d({\bf r}) \bigl[\tau({\bf r}) - \tau_{cr}({\bf r})\bigr]^q
%=K_d({\bf r}) \bigl[\rho_w\, g h({\bf r}) \sin \beta({\bf r}) -\tau_{cr}({\bf r})\bigr]^q
%\eqno (13)
\end{equation}

\noindent
where $\tau({\bf r})=\rho_w\, g h({\bf r}) \sin \beta({\bf r})$ 
[Pa] is the shear stress,
$ \beta$ [deg] is the slope angle,
$p$ and $q$ are exponents,
$ K_t({\bf r})$ [s] is the effective transport capacity coefficient,
%($[s]$ for $p=1$),
$K_d({\bf r})$ [s/m] is the effective erodibility
(detachment capacity coefficient),
%($[s/m]$ for $p=1$),
$\rho_w\, g$  is the hydrostatic pressure
of water with the unit height,
$ g=9.81$ [m/s$^2$] is the gravitational acceleration,
$\rho_w=10^3$ [kg/m$^3$] is the  mass density of water,
and
$\tau_{cr}({\bf r})$ [Pa] is the critical shear stress.
The parameters and adjustment factors for the estimation of
$D_c ({\bf r})$ and $T({\bf r})$
are functions of soil and cover properties, and their values
for a wide range of soils, cover, agricultural and
erosion prevention practices
were developed within the WEPP model \cite{wepp95}.

%The exponents $p, q$ substantially influence model behavior.
%The {\sl WEPP} model uses $q=1$ and $p=1.5$ which means that with increasing
%water flow, transport capacity increases faster than detachment.
%The comparison with the spatial extent of colluvial deposits
%in our previous study (Mitas and Mitasova, 1998)
%indicates that over long term period (decades)
%the pattern of erosion/deposition
%is closer to the results obtained with a lower value of $p$.
%This is in agreement with several studies
%which indicate that for the landscape scale modeling
%the Equations (12) and (13) are not general enough for different
%types of flow and sediment transport processes
%present in complex landscapes
%(Willgoose et al., 1989; Kirkby, 1987; Willgoose and
%Gyasi-Agyei, 1995; Mitas and Mitasova, 1998).
%Nearing et al. (1997) proposed a transport capacity equation as a
%function of stream power. The equation was re-arranged (Mitas and
%Mitasova, 1997) and implemented within the presented model.
%Its advantage is that it
%better reflects the influence of change in flow velocity
%on the net erosion/deposition.
Similarly as for the water flow equation, the steady
state sediment flow equation (\ref{eq:ssediment}) can be rewritten to include a
small diffusion term
%. First, a function representing the mass of water-carried
%sediment per unit area $\varrho({\bf r})$ in $kg/m^2$ is defined as:
%
%\begin{equation}
%\label{eq:sedmass}
%\varrho({\bf r})= \rho_s c({\bf r})h({\bf r})
%%\eqno(14)
%\end{equation}
$\propto \nabla^2 \varrho({\bf r})$:

\begin{equation}
\label{eq:difsedim}
-{\omega \over 2}\nabla^2 \varrho({\bf r})
+ \nabla\cdot [\varrho({\bf r}){\bf v}({\bf r})]
 + \varrho({\bf r}) \sigma({\bf r}) |{\bf v}({\bf r})|
= \sigma({\bf r}) T({\bf r})
%\sigma({\bf r}) T({\bf r})=Dc/T * T=Dc
%\eqno(15)
\end{equation}

\noindent
where  $\omega$ [m$^2$/s] is the diffusion constant and 
$\varrho({\bf r})= \rho_s c({\bf r})h({\bf r})$ [kg/m$^2$] is 
the mass of water-carried sediment per unit area.
 On the left hand side of the equation (\ref{eq:difsedim}), the first term
 describes local diffusion, the second term is a drift driven by the water flow
 while the third term represents a velocity dependent
'potential' acting on $\varrho({\bf r})$. The size of the diffusion
 constant is about one order of magnitude smaller than the
 reciprocal Manning's constant so that the impact of the diffusion term
is relatively small. It represents local dispersion processes
of the suspended flow, caused by microtopography
which is not captured by the digital elevation model (DEM).
The diffusion term can be modified to reflect impact of various processes.

\subsection{Numerical Solution by Path Sampling Method}

Most models of landscape processes are based on numerical solutions 
of governing partial differential equations by finite element \cite{vieux96},
finite difference \cite{saghafian96} or path sampling methods \cite{mitwrr98}. 
The path sampling method has several important advantages when compared 
with more traditional approaches. The method is very robust, can be easily 
extended into arbitrary dimension, is mesh-free and very efficient 
on parallel architectures.

Variety of path sampling methods have been explored in environmental 
applications such as simulation and transport of dissolved 
and suspended substances in water bodies \cite{mike21pa}, \cite{dimou93}, 
groundwater modeling \cite{thompson90}  and soil erosion by 
overland flow \cite{mitwrr98}. The method has important applications 
in other areas such as quantum systems where the probability 
amplitude (wavefunction) can be mapped onto a "fluid" and used 
to solve the Schrodinger equation with special symmetry
constraints. In that context, the methods are known as 
Green function Monte Carlo, diffusion Monte Carlo \cite{foulkes01}.
Equations (\ref{eq:difwater}) and (\ref{eq:difsedim}) have a similar 
form in which a linear differential operator $P$ acts on a nonnegative
unknown function $f_0({\bf r})$ (in our case, either $h({\bf r})$ or $\varrho ({\bf r})$),
  while on the right hand side, there is a
given source term $S({\bf r})$ (either $i_e({\bf r})$ or
$\sigma({\bf r})T({\bf r})$):

\begin{equation}
{P} f_0({\bf r})= S({\bf r})
\end{equation}

\noindent
Denoting by ${P}^{-1}$ the inverse operator to $P$,
 the solution can be symbolically written as:
\begin{equation}
f_0({\bf r})=  P^{-1} S({\bf r})
\end{equation}

\noindent
or explicitly, using the Green function:
\begin{equation}
f_0({\bf r})= \int_0^{\infty}\int G({\bf r},{\bf r'},t)
 {S}({\bf r'}) d{\bf r'}dt
\end{equation}

\noindent
$G({\bf r},{\bf r'},p)$
is given by the following time-dependent equation 
with an initial condition:

\begin{equation}
{\partial G({\bf r},{\bf r'},t)
\over \partial t} =-P G({\bf r},{\bf r'},t)\, ;
\;\; G({\bf r},{\bf r'},0) =\delta ({\bf r}-{\bf r'}),
%\eqno(19)
\end{equation}

\noindent
 where ${\bf r},{\bf r'}$ are positions,
$t$ is time and $\delta$ is the Dirac function. In addition,
 we assume that the spatial region is a delineated watershed with 
zero boundary condition which is fulfilled by $G({\bf r},{\bf r'},t)$. 
The corresponding equations can be interpreted 
as describing stochastic processes with 
diffusion and drift components (Fokker-Planck equations)
 and one can carry out the actual simulation of the underlying process
  using the path sampling method \cite{gardiner85}.
Our definition of the Green function effectively introduces time,
the reason being that it enables us to consider both 
time dependent and stationary phenomena/processes on the same 
footing.
Let us therefore consider the following  time-dependent 
partial differential equation:
\begin{equation}
{\partial f({\bf r},t) \over \partial t}=P f({\bf r},t)
\end{equation}
\noindent
where
\begin{equation}
P f({\bf r},t)=-\varepsilon \nabla^2 f({\bf r})
+ \nabla\cdot [f({\bf r},t){\bf v}({\bf r})]
- u({\bf r}) f({\bf r},t)
\end{equation}
so that the operator $P$ includes diffusion ($\varepsilon$ is a diffusion
constant), drift and potential (birth-decay) terms.
The potential represents a rate term
such as radioactive decay or proliferation in a chemical reaction.
Starting from some initial $f({\bf r},0)$ the solution of this equation
can now be written as

\begin{equation}
\label{eq:gensolution}
f({\bf r},t)=\exp(-tP) f({\bf r},0)
\end{equation}

\noindent
This solves the given differential equation 
as can be verified by taking the derivative
according to $t$. 

%---skip this?
%
%Similarly, for a stationary equation with a source
%
%\begin{equation}
%Pf_s({\bf r})=s({\bf r})
%\end{equation}
%we can write (since solution is time independent):
%
%\begin{equation}
%f_s({\bf r})=P^{-1} s({\bf r})
%
%f_s({\bf r})=\int_0^{\infty} {\rm exp}(-tP) s({\bf r})dt
%\end{equation}
%The only task now is to find ${\rm exp}(-tP)$ and apply it to the 
%particular function $f({\bf r},0)$ or $s({\bf r})$.

For solving our transport equations (\ref{eq:gensolution}),
  (\ref{eq:difwater}) and  (\ref{eq:difsedim}),
we need the Green function (matrix element)
$G( {\bf r},  {\bf r'},t)$
%=\langle {\bf r}|\exp(-tP)|{\bf r'} \rangle$
for arbitrary ${\bf r},{\bf r'}$ so that we can write

\begin{equation}
f({\bf r},t) = 
\int G( {\bf r},  {\bf r'},t) f({\bf r'},0)d{\bf r'}
%= \int \langle {\bf r}|\exp(-tP)|{\bf r'} \rangle f({\bf r'},0)d{\bf r'}
\end{equation}
Suppose that we know the Green function for some time slice $\tau$.
The solution at abitrary  multiple of $\tau$ can be found by
iteration, e.g.,
the solutions at time $\tau$ and $2\tau$ are given by

\begin{equation}
f({\bf r},\tau) = 
\int G({\bf r}, {\bf r'},\tau) f({\bf r'},0) 
%\langle {\bf r}|\exp(-\tau P)|{\bf r'} \rangle f({\bf r'},0)
d{\bf r'}
\end{equation}

\begin{equation}
f({\bf r},2\tau) = 
\int G({\bf r}, {\bf r'},\tau) 
%\langle {\bf r}|\exp(-\tau P)|{\bf r'} \rangle 
f({\bf r'},\tau)d{\bf r'}
\end{equation}
etc.
%This can be iterated arbitrary number of times enabling us to obtain the solution
%for any given time. 
%(a) find the matrix element ofr a given $\tau$, 
%(b) explain how to represent function $f(r,t)$ as a density of sampling points
%(c) find how to propagate the sampling points by a time step 
%$\tau$ so that we get the solution by iteration.
%First, we will find the matrix elements. Suppose that $\tau$ is small and 
%$P$ represents only a diffusion $P -d_0 \nabla^2$, then
%<{\bf r}|{\rm exp}(-\tau P)|{\bf r'}>=C_N {\rm exp}[-|{\bf r'}-{\bf r}|^2/4\tau
%\end{equation}
%which is well known result - solution of the diffusion equation (its Green's function)
%is a gaussian. If we add a constant drift ${\bf v}_0$ the matrix element is
%
%%\begin{equation}
%<{\bf r}|{\rm exp}(-\tau P)|{\bf r'}>=C_N {\rm exp}[-|{\bf r'}-{\bf r}i-\tau{\bf v}_0|^2/4\tau
%\end{equation}
%so the solution of a diffusion with a constant drift is a gaussian with drifting center.
%If we include laso $u(.)$ term (potential) we get
If the time slice $\tau$ is small, using the Trotter-Suzuki formula
one finds
\begin{equation}
G({\bf r}, {\bf r'},\tau)=
%\langle{\bf r}|{\rm exp}(-\tau P)|{\bf r'}\rangle=
C_N {\rm exp}[-|{\bf r'}-{\bf r}-\tau{\bf v}_0|^2/4\tau]
{\rm exp}[-\tau(u({\bf r'})+u({\bf r'})] + O(\tau^3)
\end{equation}
where $C_N$ is the normalization constant of the gaussian.
Suppose now that we represent the function 
$f({\bf r},t)$ as a density of sampling points (also called walkers)

\begin{equation}
f({\bf r},0)\;\longrightarrow \;\sum_{m=1}^M \delta({\bf r}-{\bf r}_m)
\end{equation}
assuming that the density can be easily estimated by a histogram.
%This can be done by creating a histogram - dividing the space into
%bins and computing the number of points in each bin. 
By decreasing the size of histogram bins and
increasing the number of walkers one can get 
arbitrarily accurate approximation of a given function.
If we rescale the function $f({\bf r},0)$ by a constant, 
then the weight of the delta functions must change as well,
 so we introduce a walker weight $w_m$
which gives the weight of each walker contribution to the bin.
Therefore each walker is specified by its position and weight.
%Third, we need to find a way how to propagate a single walker. 
By substituting for  
 $f({\bf r},0)$ the set of walkers $ \Sigma_m w_m\delta({\bf r}-{\bf r}_m)$ 
to the equation above and by carrying out
the integration we get for the walker with label $m$ the following
expression 
\begin{equation}
\int G({\bf r}, {\bf r'}, t)
%\langle{\bf r}|\exp(-\tau P)| {\bf r'}\rangle 
w_m\delta({\bf r'}-{\bf r}_m)d{\bf r'}=
C_N w_m \exp[-|{\bf r}_m-{\bf r}-\tau{\bf v}_0|^2/4\tau]
\exp[-\tau(u({\bf r'})+u({\bf r_m}))/2]
\end{equation}
which is a gaussian with renormalized weight. 
In order to carry out the next iteration 
in the same manner, we need to restore the delta-function/walker
representation.  This is done 
by {\em sampling} the gaussian - finding a new position of the walker
by drawing a random vector from a gaussian distribution around ${\bf r}_m$,
drifting by $\tau{\bf v}_0$ and 
updating the weight by the renormalization factor. 
Therefore
 the new position is given by
\begin{equation}
{\bf r}_m^{new}={\bf r}_m + \tau {\bf v}_0 + {\bf g}
\end{equation}
where ${\bf g}$ is a random vector with gaussian components with
variance $\tau$ 
 while the updated weight reads
\begin{equation}
w_m^{new}=w_m \exp[-\tau(u({\bf r}_m^{new})+u({\bf r}_m))/2]
\end{equation}

\noindent
The walker representation is based on duality between the particle and field
representation of spatially distributed phenomena.
Within this concept, density of particles in
space defines a field and vice versa, ie, field is represented by
particles/samples with corresponding spatial distribution. Using
this duality, processes can be modeled as evolution of fields or evolution
of spatially distributed particles as described above.

The solution is then described as a function with statistical error 
proportional to $1/\sqrt{M}$ where $M$ is the number of walkers. 
The solution for steady state can be obtained in two ways. One possibility
is to evolve the time-dependent solution until the steady state is reached.
Statistically, this is less efficient since the equilibration 
part of walker paths are thrown away. The second option is to start
from the initial walker distribution proportional to the source 
$S({\bf r)}$ which is then evolved as given above. The steady
state solution is obtained by accumulation of the evolving source over
the relevant period (ie. until all the walkers flow out or die out),
 effectively performing the corresponding integral over time.
%\paragraph{Water flow model extensions for depressions, preferential flow,
%shallow channels, semipermeable barriers, vegetative buffers}
%Several effects that require dynamic solution can be approximated
%within the current framework if certain conditions are fulfilled.

The accumulation process can be also interpreted as an approximation of
a
dynamical solution for shallow water flow, in which
velocity is mostly controlled by terrain slope and surface
roughness rather than by water depth and friction slope, and therefore its
change over time at a given location is negligible.
The robustness of the path sampling method enables the use of a very wide 
range of input data and complex conditions that can be modelled without 
manually editing the input data, as it is common with traditional methods. 
It can be therefore used efficiently for exploration of large number of 
landscape configurations, needed for various applications, including conservation
planning and sediment control.
%To extend the water flow model applicability to terrains with
%large depressions or areas with undefined gradient,
%a spatially variable diffusion can be used.
%By defining the diffusion term $\varepsilon({\bf r})$
%as a function of water depth and the
%velocity of flow as a function of an approximate water flow momentum,
%water fills the depressions or spreads in the flat area
%and then flows out in the prevailing flow direction.
%The model assumes spatially variable steady infiltration rate
%that is subtracted from the rainfall to generate runoff as input field. 
%To improve the modeling of effects of buffers, steady, spatially
%variable infiltration rate has been included also for flowing water
%assuming that numerical evolution sufficiently approximates
%the dynamics.
%Full dynamic solution will enable us to expand these effects 
%to more complex situations with greater range of water flow depth.

%Sometimes the DEM in spite of its high resolution is not detailed or
%accurate enough to capture features that control flow, such as channels,
%ditches, pipes, etc - in such a case the gradients obtained
%from the DEM are combined with the gradients that define the channel/ditch etc.

%We can obtain an approximation of dynamic wave solution for
%the following simplified conditions:
%velocity does not change in time (we considered time averaged
%velocity?) and is dependent on slope and unit water depth
%we can interpret the solution as unsteady/evolving/timevarying flow
%for steady rainfall excess and approximation of diffusion wave solution.


\paragraph{Multiscale formulation using nested grids}

Both {\it spatially variable accuracy} and {\it resolution} can be
implemented by reformulating the  solution through the Green function
given by Equation (18).
The integral Equation (18) can be multiplied by a {\it reweighting }
 function  $W({\bf r})$:

\begin{equation}
W({\bf r}) f_0({\bf r})= \int_0^{\infty}\int 
W({\bf r})G({\bf r},{\bf r'},t)
  S({\bf r'})  d{\bf r'}dt =
 \int_0^{\infty}\int G^*({\bf r},{\bf r'},t)
 S({\bf r'})  d{\bf r'}dt
%\eqno(25)
\end{equation}
 which is equal to the appropriate increase in accuracy ($W({\bf r})>1$)
 in the regions of interest while it is
  unity elsewhere. The function $W({\bf r})$ can change (abruptly or smoothly)
  between regions with unequal
 resolutions  and in fact, can be optimally adapted to the quality of
 input data (terrain, soils, etc) so that the accurate solution
 is calculated only in the regions with correspondingly accurate inputs.
 The reweighted Green function $G^*({\bf r},{\bf r'},t)$, in effect,
  introduces higher density of sampling points in the region with
 large $W({\bf r})$. The statistical noise will be spatially
 variable as
    $\approx 1/[W({\bf r})\sqrt{M}]$, where $M$ is the average number
 of samples  resulting in the accuracy increase for the areas
 with $W({\bf r}) > 1$.
The multiscale approach was presented for modeling with spatially variable 
accuracy and homogeneous resolution in \cite{mitvienna98} and
for spatially variable resolution in \cite{mitlle01}.

\section{GIS IMPLEMENTATION AND APPLICATION}

To increase the efficiency in data preparation and results analysis 
the method was integrated within GRASS GIS. Its functionality is being tested 
in several locations in the North Carolina Triangle area, with ongoing
monitoring that will support the specific model calibration and validation
for various conditions.

\subsection{Implementation in GIS}

The path sampling approach can be used to simulate a wide range of fluxes
described by continuity equations. The algorithm that propagates the walkers
was therefore implemented as a library that can be used to build
models for other types of transport. This library was then used to
build two specific modules for simulation of water flow
and for sediment transport and erosion/deposition.
Both modules are fully integrated with GRASS GIS
and are executed from within the system. A wide range of
GRASS GIS tools is used to preprocess the georeferenced input data as well as to
analyze and visualize the outputs.

The module {\bf r.sim.water} solves the equation (\ref{eq:difwater})
and computes maps representing spatial distribution of
steady state water depth and discharge. Optionally, it is possible to output
time series of raster maps representing evolution of water flow until it
reaches steady state.
Input data include raster maps representing elevation, first-order partial 
derivatives of elevation surface, rainfall excess, 
%(water available for runoff, e.g., rainfall-infiltration-vegetation intercept [mm/hr]) 
and  Manning's surface roughness coefficient.  
Elevation surface gradient can be combined with gradients representing 
channels or other features that control water flow.
There is a number of parameters that allow the user to control
the simulation by adjusting the number of walkers, number of iterations and diffusion,
including a modified diffusion term which enables
to overcome elevation depressions or obstacles when accumulated water depth exceeds
a threshold water depth value. Number of iterations effectively 
controls the duration of simulated event.
% it is reached, diffusion term
%increases as given by  halpha and advection term (direction of flow)
%is given as "prevailing" direction of flow computed as average of flow
%directions from the previous hbetanumber of grid cells.
%Simulation error can be analyzed using err raster file add exact definition
%- the resulting water depth is an average, err is its RMSE ?). 
It is also possible to output series of site maps representing 
spatial distribution of walkers at different simulation times.

The module {\bf r.sim.erosion} solves the equation (\ref{eq:difsedim}) and  computes 
maps representing spatial distribution of steady state sediment flow rate,
sediment concentrations and net soil erosion/deposition rate.
Input data include raster maps representing elevation, first-order partial 
derivatives of elevation surface, water depth (computed e.g. by {\bf r.sim.water}), 
detachment capacity coefficient, transport capacity coefficient, critical shear stress and 
Manning's surface roughness coefficient. Optionally, it is possible to output 
transport capacity map and transport capacity-limited erosion/deposition rates.
Time series of raster maps representing evolution of sediment flow and erosion/deposition
for the given water depth can also be produced.
Similarly as for water, the simulation can be controlled by
 adjusting the number of walkers, number of iterations and a diffusion term.

The module {\bf r.terradyn} is a terrain evolution routine
for small watersheds. Currently, it is implemented as a shell script run within GRASS
GIS that solves equation (\ref{eq:elevchange}), producing a time series of updated
elevations based on the distributed erosion/deposition rates from {\bf r.sim.sediment}.
At each iteration, the terrain change is smoothed via the GRASS GIS
module {\bf r.neighbors}.  Even with the
additional diffusion term in (\ref{eq:elevchange2}), this step is critical for
the suppression of numerical noise over a large number of iterations.
Partial derivatives of the elevation surfaces are computed using the module {\bf r.slope.aspect}.
The user can define the same control variables and spatially distributed input raster files
as with {\bf r.sim.water} and {\bf r.sim.sediment}, as well as the number of terrain update
iterations and the smoothing parameter.  Currently, the bulk density $\varrho_b$ and the
gravitational diffusion coefficient $\gamma$ are user-definable scalars; however,
since these parameters should be a function of soil type, we plan to implement them
as distributed variables in the future. In addition, the user can opt to dynamically 
redefine the transport and detachment coefficients and the critical shear stress 
based on the water depth and/or depositional history.

\subsection{Application in North Carolina}

The model is currently being tested at several locations in North Carolina
with ongoing changes in land use and monitoring of water flow and sediment
transport.
%\paragraph{Centennial Campus}
%http://skagit.meas.ncsu.edu/~helena/wrriwork/cenntenial/gim*
The first test area is at the Southwest section of
the North Carolina State University Centennial Campus 
that is being transformed from forest and meadows to
developed area with educational and recreational facilities (Figure~\ref{fig:cc}).
To reduce the negative effects of construction on soil
erosion and water pollution, sediment control measures need
to be installed. The presented example explores the use of GIS
and simulations to better assess the need for sediment control
measures and plan their most effective locations.
%The water and sediment flow at three $~30$ acre subwatersheds (A, B, C),
%were monitored at locations close to their outlets to support
%the model calibration and validation.
During the school construction, 3 large checkdams 
were installed to control sediment. In near future, the construction of 
golf course will lead to removal of a large portion of the current forest.
To simulate the impact of this development with 
conservation measures, several landscape models were created, representing
the current and planned elevation surfaces and various configurations of landcover
including: (a) current state, (b) start of construction without control measures, 
(c) construction with extended buffers, (d) staged construction, 
(f) finished grading before planting, (g) finished golf course. 
Water and sediment flow as well as 
erosion/deposition were then simulated for each configuration 
for a 42mm/hr, steady rainfall for a duration of 1 hour.
The results are illustrated by Figure~\ref{fig:cc} and Table~\ref{tab:cc}.
As expected, the simulations demonstrate that the construction will 
lead to substantial increase in runoff and sediment transport. Surprisingly,
the major negative impact is predicted outside the actual
disturbed area, in the form of erosion due to concentrated flow within the preserved
mandatory buffers. The channels need to be surveyed prior and during the construction
to verify this result. Relative efficiency of different conservation measures
was quantified by comparing the total runoff, sediment yield and total 
erosion for two types of extended buffers and
staged construction.

\begin{figure}[Htb!]
\begin{center}
\includegraphics[height=0.78\textheight, width=\textwidth]{ccsimwefin.eps}
\caption{Spatial pattern of land cover (a,b),
overland flow (c,d) and net erosion/deposition (e,f)
before and at the start of construction. The largest increase in erosion
is predicted within the protective buffer due
to increased water flow. Simulations also  show the effectiveness
of large checkdams in controlling water flow from the parking lot.}
\label{fig:cc}
\end{center}
\end{figure}

The simulations show that the standard buffers provide very limited protection
(Figure~\ref{fig:cc}); however, their extension into most of the areas with concentrated 
flow, using the cover that preserves high infiltration and vegetation intercept,
 substantially reduces the negative impact. The effectiveness of 
staged construction varied depending on the size and location of the disturbed 
area, it was not very effective when entire "half-watershed" that included
several concentrated flow areas was disturbed at once (Figure~\ref{tab:cc}, stage cs1).

\begin{table}[htb]
\caption{Discharge and sediment yield at the oulet of the subwatershed 
with the planned golf course, and total erosion rate for the entire study area (100acres)
at different stages of development and conservation measures.
%Simulation for 1hr, 42mm/hr steady rainfall, representing a 2 year design storm. 
Discharge and sediment yield equal to zero reflect the impact of numerous 
depressions in the modeled subwatershed for stages that produce lower runoff.}
\label{tab:cc}
\newcommand{\m}{\hphantom{$-$}}
\newcommand{\cc}[1]{\multicolumn{1}{c}{#1}}
\renewcommand{\tabcolsep}{1.8pc} % enlarge column spacing
%\renewcommand{\arraystretch}{1.2} % enlarge line spacing
\begin{tabular}{@{}lllll}
\hline
land use                  & discharge  & sediment yield & erosion rate \\
\quad                     & [m$^3$/s]  & [kg/s]         &  [kg/s]      \\
\hline
pre-development           & \m0.0  & \m0.0   & \m84. \\
current                   & \m0.0  & \m0.0   & \m87. \\
start of construction     & \m0.46 & \m0.26  & \m968. \\
extended buffers grass    & \m0.45 & \m0.25  & \m337.  \\
extended buffers forest   & \m0.0  & \m0.0   & \m142.  \\
staged constr. section 1  & \m0.30 & \m0.18  & \m771.  \\
staged constr. section 2  & \m0.16 & \m0.11  & \m216.  \\
%final grading            & \m0.04 & \m0.17  & \m3.955 \\
%finished golfcourse      & \m0.25 & \m0.43  & \m4.846 \\
\hline
\end{tabular}\\[2pt]
\end{table}



%test the dynamic version with dynamic rainfall excess full coupling of sediment 
%calibration and validation is a complex issue (REFERENCES) - data measured at
%one or several points can be obtained by quite different combinations
%of input parameters and spatially distributed data for overland
%flow and erosion deposition are difficult to obtain at landscape scale
%also up-to date data are needed
%data are imporving rapidly - 13 real time stations in Charlotte,
%ground based lidar, 

%The second testing location is the small watershed within
%the North Carolina State University's experimental farm where the Sediment
%and Erosion Control Research and Education facility is located (Figure 1).
%Currently, this area is being used to calibrate and verify {\bf r.terradyn}.
%Elevation and land cover raster files
%were created for this region at 2 m resolution and rainfall rates
%were obtained from the Lake Wheeler Road State Climate Office field station.
%The resulting distributed infiltration and Manning's coefficient rasters serve as
%inputs into the {\bf r.sim.water}, {\bf r.sim.sediment}, and {\bf r.terradyn}
%simulations, with other input parameters being varied over a wide range
%of values to determine the scope of application and model stability.
%In Figure~\ref{fig:terrdynlw}, the rainfall excess, Manning's coefficient, critical
%shear, and the detachment and transport coefficients are static scalar values,
%with the diffusion coefficient set to XX and the bulk density to 0.4.
%Gulleys and channels develop which have been observed on site at the location.
%Although calibration and verification are ongoing, early {\bf r.terradyn} results
%show promise for verifiable elevation modification as well as grain size dependent
%large scale pattern formation. Future work includes the development of field
%measurements of water and sediment discharge rates and medium time scale terrain changes.
%We also plan to focus on the simulation of various combinations of erosion control measures,
%allowing modellers to optimize a full network of control measures for a given
%disturbed watershed so as to minimimize sediment discharge and terrain change impact.

%\begin{figure}[htb]
%\begin{center}
%\includegraphics[width=\textwidth]{terrdynlw.eps}
%%[width=\textwidth]{ccsimwefin.eps}
%\caption{Initial elevation surface and the developed rills and gullies
%after 10 iterations.}
%\label{fig:terrdynlw}
%\end{center}
%\end{figure}


\section{CONCLUSIONS}

High resolution, spatially distributed simulations provide new insight
into the spatial aspects of impacts of disturbances in complex landscapes.
Integration with GIS supports efficient design and evaluation of various
configurations of conservation measures providing valuable information
for erosion prevention and sediment control. The path sampling method 
provides the robustness necessary for simulating diverse landscapes
and complex interactions. Future development, based on the combination
of finite difference and path sampling method, is focusing on expanding
the capabilities for more realistic modeling of dynamics of water and sediment flow
and terrain evolution. Routine applications will require comprehensive
model calibration and validation that will be based on the established
and new monitoring and experiments at sites in North Carolina.


\begin{thebibliography}{9}

\bibitem{answers91}                                         
C.C. Rewerts and B.A. Engel, ASAE Paper No.91-2621 (1991).

\bibitem{swat94}
R. Srinivasan and J.G. Arnold, Water Resources Bulletin 30 (1994) 453.

\bibitem{saghafian96}
B. Saghafian, in: GIS and Environmental Modeling: Progress and Research Issues,
M.F. Goodchild, L.T. Steyaert and B.O. Parks (eds.), GIS World Inc. (1996) 205.

\bibitem{vieux96}
B.E. Vieux, N.S. Farajalla and N. Gaur, 
in: GIS and Environmental Modeling: Progress and Research Issues,
M.F. Goodchild, L.T. Steyaert and B.O. Parks (eds.), GIS World Inc. (1996) 199.

\bibitem{grassbook}
M. Neteler and H. Mitasova, Open Source GIS: A GRASS GIS Approach, 
Kluwer Academic Press, Boston, 2002.  

\bibitem{siberia95}
G.R. Willgoose, and Y. Gyasi-Agyei, 
in: Proceedings of the APCOM XXV Conference, Brisbane (1995) 555.

\bibitem{casc2d95}
P.Y. Julien, B. Saghafian, and F.L. Ogden, 
Water Resources Bulletin 31 (1995) 523.

\bibitem{child01}
G. Tucker, S. Lancaster, N. Gasparini and R. Bras, 
in: Landscape Erosion and Evolution Modeling, R.S. Harmon
and W.W. Doe (eds.), Kluwer, New York (2001) 349.

\bibitem{mitwrr98}
L. Mitas and H. Mitasova, Water Resources Research 34 (1998) 505.

\bibitem{moorefoster90}
I.D. Moore and G.R. Foster, 
in:  Process Studies in Hillslope Hydrology, 
M.G. Anderson and T.P. Burt (eds.), John Wiley (1990) 215.

\bibitem{ding84}
S.L. Dingman, Fluvial hydrology, Freeman, New York, 1984.

\bibitem{karambas02}
T.V. Karrambas and C. Koutitas, 
J. of Waterway, Port, Coastal and Ocean Engineering 128 (2002) 102.

\bibitem{parker00} 
G. Parker, C. Paola and S. Leclair,
J. of Hydraulic Engineering, 126 (2000) 818.


\bibitem{paola95} 
C. Paola and R. Seal, Water Resources Research, 31 (1995) 1395.

\bibitem{dietrich93}
W.E. Dietrich, C.J. Wilson, D.R. Montgomery, J. McKean, 
J. of Geology, 101 (1993) 259.

\bibitem{horikawa88}
K. Horikawa, Nearshore dynamics and coastal processes, University Tokyo Press,
Tokyo, 1988.

\bibitem{haan94}
C.T. Haan, B.J. Barfield and J.C. Hayes, 
Design Hydrology and Sedimentology for Small Catchments,
Academic Press (1994) 242.

\bibitem{govin91}
R.S. Govindaraju and M. L. Kavvas, 
J. of Hydrology 127 (1991) 279.

\bibitem{fostermeyer72}
G.R. Foster and L.D. Meyer, 
in: Sedimentation: Symposium to Honor Prof. H.A. Einstein,
 H. W. Shen (ed.), Colorado State University, Ft. Collins, CO (1972), 12.1.

\bibitem{wepp95}
D.C. Flanagan and M.A. Nearing (eds.),
USDA-Water Erosion Prediction Project,
 NSERL report no. 10, National Soil Erosion Lab.,
USDA ARS, Laffayette, IN, 1995.


\bibitem{mike21pa}
Danish Hydrologic Institute, MIKE21 PA, http://www.dhisoftware.com/ (2003).

\bibitem{dimou93}
K.N. Dimou and E.E. Adams, 
Estuarine, Coastal and Shelf Science 33 (1993) 99.

\bibitem{thompson90}
A.F.B. Thompson and L.W. Gelhar, 
Water Resources Research 26 (1990) 2541.

\bibitem{foulkes01} 
M.W.C. Foulkes, L. Mitas, R.J. Needs and G. Rajagopal, 
Rev. Mod. Phys. 73 (2001) 33.

\bibitem{gardiner85}
C.W. Gardiner, Handbook of Stochastic Methods for Physics,
Chemistry, and the Natural Sciences, Springer, Berlin, 1985.

\bibitem{mitvienna98}
L. Mitas and H.  Mitasova, 
in: Modelling Soil Erosion, Sediment Transport and Closely Related Hydrological Processes,
W.Summer, E. Klaghofer and W.Zhang (eds.), IAHS Publication no. 249 (1998) 81.


\bibitem{mitlle01}
H. Mitasova and L. Mitas, 
in: Landscape erosion and landscape evolution modeling, Harmon R. and Doe W. (eds.), 
Kluwer, New York (2001) 321.

\end{thebibliography}

%\bibitem{Scho70} S. Scholes, Discuss. Faraday Soc. No. 50 (1970) 222.
%\bibitem{Mazu84} O.V. Mazurin and E.A. Porai-Koshits (eds.),
%                 Phase Separation in Glass, North-Holland, Amsterdam, 1984.
%\bibitem{Dimi75} Y. Dimitriev and E. Kashchieva, 
%                 J. Mater. Sci. 10 (1975) 1419.
%\bibitem{Eato75} D.L. Eaton, Porous Glass Support Material,
%                 US Patent No. 3 904 422 (1975).
%
%References should be collected at the end of your paper. Do not begin
%them on a new page unless this is absolutely necessary. They should be
%prepared according to the sequential numeric system making sure that
%all material mentioned is generally available to the reader. Use
%\verb+\cite+ to refer to the entries in the bibliography so that your
%accumulated list corresponds to the citations made in the text body. 
%
%Above we have listed some references according to the
%sequential numeric system \cite{Scho70,Mazu84,Dimi75,Eato75}.

%\begin{table}[htb]
%\caption{The next-to-leading order (NLO) results
%{\em without} the pion field.}
%\label{table:1}
%\newcommand{\m}{\hphantom{$-$}}
%\newcommand{\cc}[1]{\multicolumn{1}{c}{#1}}
%\renewcommand{\tabcolsep}{2pc} % enlarge column spacing
%\renewcommand{\arraystretch}{1.2} % enlarge line spacing
%\begin{tabular}{@{}lllll}
%\hline
%$\Lambda$ (MeV)           & \cc{$140$} & \cc{$150$} & \cc{$175$} & \cc{$200$} \\
%\hline
%$r_d$ (fm)                & \m1.973 & \m1.972 & \m1.974 & \m1.978 \\
%$Q_d$ ($\mbox{fm}^2$)     & \m0.259 & \m0.268 & \m0.287 & \m0.302 \\
%$P_D$ (\%)                & \m2.32  & \m2.83  & \m4.34  & \m6.14  \\
%$\mu_d$                   & \m0.867 & \m0.864 & \m0.855 & \m0.845 \\
%$\mathcal{M}_{\mathrm{M1}}$ (fm)   & \m3.995 & \m3.989 & \m3.973 & \m3.955 \\
%$\mathcal{M}_{\mathrm{GT}}$ (fm)   & \m4.887 & \m4.881 & \m4.864 & \m4.846 \\
%$\delta_{\mathrm{1B}}^{\mathrm{VP}}$ (\%)
%                          & $-0.45$ & $-0.45$ & $-0.45$ & $-0.45$ \\
%$\delta_{\mathrm{1B}}^{\mathrm{C2:C}}$ (\%)
%                          & \m0.03  & \m0.03  & \m0.03  & \m0.03  \\
%$\delta_{\mathrm{1B}}^{\mathrm{C2:N}}$ (\%)
%                          & $-0.19$ & $-0.19$ & $-0.18$ & $-0.15$ \\
%\hline
%\end{tabular}\\[2pt]
%The experimental values are given in ref. \cite{Eato75}.
%\end{table}

\end{document}
