%% Edited Book Template file, EdbkTmpl.tex, 
%% LaTeX2e 
%% Uncomment documentclass, 
%\documentclass{KapEdbk} % Computer Modern font calls
\documentclass{kapedbk} % Computer Modern font calls

%% If you use a font encoding package, please enter it here, i.e.,
%  \usepackage{T1enc}
\usepackage{epsfig}

% PostScript font calls
% If you use the EdbkPs PS font file, you may need to edit it
% to make sure the font names match those on your system. See
% the top of the EdbkPs.sty file for more info.
% \usepackage{EdbkPs} 

%% How many levels of section head would you like numbered?
%% 0= no section numbers, 1= section, 2= subsection, 3= subsubsection
%%==>>
\setcounter{secnumdepth}{3}
%\setcounter{secnumdepth}{1}
\setcounter{chapter}{10}
\pagenumbering{arabic}
\setcounter{page}{41}
\setcounter{figure}{0}
%\turnoff
\font\ninerm=cmr9


%% How many levels of section head would you like to appear in the
%% Table of Contents?
%% 0= chapter titles, 1= section titles, 2= subsection titles, 
%% 3= subsubsection titles.
%%==>>
\setcounter{tocdepth}{3}

%%% Uncomment one of the Following:
\kluwerbib
%\normallatexbib

%%%%%%% Author and Topic Indices
%% If you want to have both an author and a topic index, uncomment this:
%\startauthorindex

%%%% <<==  End Formatting Commands You Can Set or Change %%%%%%%%%%%%%%%%%
%%%%%%%%%%%%%%%%%%%%%%%%%%%%%%%%%%%%%%%%%%%%%%%%%%%%%%%%%%%%%%%%%%%%%%%%%


\begin{document}
\articletitle{MULTISCALE SOIL EROSION SIMULATIONS FOR LAND USE MANAGEMENT}

%% optional, to supply a shorter version of the title for the running head:
\chaptitlerunninghead{\bf Multiscale Soil Erosion Simulations}

\author{Helena Mitasova$^1$ and Lubos Mitas$^2$}
\affil{$^1$ University of Illinois at Urbana-Champaign;
$^2$ North Carolina State University}

%------------ body of article ------------------->>
\bigskip
\bigskip
\bigskip
\noindent
\section{INTRODUCTION}

Increasing pressures on the land and an improved understanding of human
impacts on the environment are leading to profound changes in land management,
with emphasis on integration of local actions with watershed-scale approaches.
This trend has a significant impact on the development of supporting {\sl GIS}
and modeling tools. Complex, distributed, physics-based models
are needed to improve understanding and prediction of landscape
processes at any point in space and time. At the
same time, land owners and managers working in the watersheds and
fields need fast and easy to use models for which the input data are readily available.

Recent advances in Geographic Information Systems ({\sl GIS}) technology and
linkage of numerous models with {\sl GIS} (e.g., Wilson and Lorang, 1999;
Moore et al., 1993; Doe et al., 1996; Saghafian, 1996;
Srinivasan and Arnold, 1994; Vieux et al., 1996; Johnston and Srivastava, 1999)
have created a potential to develop an environment
for coordination of conservation efforts at different management
levels.
% by providing tools to design and evaluate the impact of land
%use alternatives at both local and watershed levels for a given
%time horizon. 
These advances facilitate evaluation of prevention practices
based not only on the type of the prevention measure, but also on their
location within the watershed. Spatial analysis and simulation can also
provide supporting information for allocation of resources to those areas
and those types of practices which will provide the most effective protection.

To reflect the need for modeling at different levels of
complexity, from fast, approximate estimates for risk assessment
to more detailed simulations for predictions and
land use design, a set of models with increasing complexity was developed
(Mitas and Mitasova, 1998; Mitasova et al., 1999).
The simple models {\sl RUSLE3D} 
(Revised Universal Soil Loss Equation for Complex Terrain)
 and {\sl USPED} (Unit Stream Power-based Erosion Deposition) are based on
modifications of well established equations representing
special cases of erosion regimes. The basic empirical parameters for these models
are available, however their applicability to a wide
range of conditions is limited. The new distributed,
process-based model {\sl SIMWE} (SIMulated Water Erosion)
provides capabilities to simulate more complex effects, however
both the experimental and theoretical research are still very active
and underlying equations, as well as the input parameters,
are under continuing development.

This chapter presents theoretical basis for the distributed, process-based
{\sl SIMWE} model, describes the relation between the processes modeled by
 {\sl SIMWE} and the {\sl RUSLE3D} and {\sl USPED} models and
illustrates the use of the presented models
for evaluation and design of different conservation measures.

\bigskip
\noindent
\section{METHODS}

To model spatio-temporal distribution of sediment transport and 
erosion/deposition at any point and time, 
a complex system of interacting processes has
to be simulated, including rainfall events, vegetation growth,
surface, subsurface and ground water flow, soil detachment, transport
and deposition. Excellent examples of continuous time
simulation systems, which integrate a wide range of interacting
processes important for land use management
 are {\sl WEPP} (Flanagan et al., this volume), {\sl LISEM} 
(Jetten and de Roo, this volume) or
 {\sl SWAT} (Srinivasan and Arnold, 1994).
Spatial components of these systems are usually based
on 1D routing of water and sediment through
homogeneous hydrologic units (e.g., hillslope segments, subwatersheds),
limiting the range of spatial effects that can be simulated.
To more accurately capture the impacts of spatially
variable conditions a new generation of hydrologic and sediment 
transport models introduced 2D flow routing capabilities  
({\sl SIBERIA}: Willgoose and Gyasi-Agyei, 1995; 
{\sl CASC2d}: Julien et al., 1995; Ogden and Heilig, this volume;
{\sl CHILD}: Tucker et al., this volume; 
{\sl SIMWE}: Mitas and Mitasova, 1998).
{\sl SIMWE} uses spatially continuous approach to
modeling of erosion and deposition 
with the modeled phenomena represented
by {\sl multivariate functions} and the
flow of water and sediment described as {\sl multivariate vector fields} rather
than systems of 1D flows.  The model was developed for
subsystems describing overland water and sediment flow with focus on
simulations of phenomena important for land use management.
The concept can be extended to other processes, including the 3D subsurface
flows.

\bigskip \medskip \noindent
%{\bf 2.1 Process-based Overland Water and Sediment Flow Model}
\subsection{Process-based Overland Water and Sediment Flow Model}

\medskip
The spatially continuous approach uses a bivariate form of continuity
equations to describe water flow and sediment transport over a complex
terrain with spatially variable rainfall excess, land cover and soil
properties during a rainfall event.

\medskip
\noindent
%{\bf 2.1.1 Shallow overland flow}
\subsubsection{Shallow overland flow}

\quad

\medskip
Shallow overland water flow is described by the bivariate form of the St Venant
 equations  (e.g., Julien et al., 1995).  The continuity relation is given by:
\leftequation
{\partial h({\bf r},t) \over \partial t} =
 i_e({\bf r},t) - \nabla \cdot {\bf q}({\bf r},t)
%\eqno (1)
\endleftequation

\noindent
while the momentum conservation in the diffusive wave approximation
has the form:

\leftequation
{\bf s_f}({\bf r},t)={\bf s}({\bf r})-\nabla h({\bf r},t)
%\eqno (2)
\endleftequation

\noindent
 where ${\bf r}=(x,y)$ in $m$ is the position,
$t$ in $s$ is the time,
$h({\bf r},t)$ in $m$ is the depth of overland flow,
$i_e({\bf r},t)$ in $m/s$ is the
rainfall excess = rainfall $-$ infiltration,
${\bf q}({\bf r},t)$ in $m^2/s$ is the unit flow discharge
(water flow per unit width),
${\bf s}({\bf r})= - \nabla z({\bf r})$ is the negative elevation
gradient,
$z({\bf r})$ in $m$ is the elevation,
and ${\bf s_f}({\bf r},t)$ is the negative gradient of
overland flow surface (friction slope).

For a shallow water
overland flow, with the hydraulic radius approximated
by the normal flow depth $h({\bf r},t)$ (Moore and Foster, 1990),
the unit discharge is given by:

\leftequation
{\bf q}({\bf r},t)={\bf v}({\bf r},t) h({\bf r},t)
%\alpha h^m {{\nabla z({\bf r})} \over {|\nabla z({\bf r})|}}
%\eqno(3)
\endleftequation

\noindent
where ${\bf v}({\bf r},t)$  in $m/s$ is the flow velocity.
The system of equations (1-3) is closed
using the Manning's relation between $h({\bf r},t)$ and
 ${\bf v}({\bf r},t)$

\leftequation
{\bf v}({\bf r},t)=
{C\over n({\bf r})} h({\bf r},t)^{2/3}
|{\bf s_f}({\bf r})|^{1/2} {\bf s_{f0}({\bf r})}
%\eqno(4)
\endleftequation

\noindent
where $n({\bf r}) $ is the dimensionless Manning's coefficient, $C=1$
is the corresponding dimension constant  in $m^{1/3}/s$ (Dingman, 1984),
and ${\bf s_{f0}}({\bf r})={\bf s_f}({\bf r})/|{\bf s_f}({\bf r})|$
is the unit vector in the friction slope direction.
 To account for spatially variable cover necessary for
land use management, we consider $n({\bf r})$
as  explicitly location dependent.

In this chapter, the solution of continuity and
momentum equations for a steady state $\partial h({\bf r},t)/\partial t =0$,
is considered to be an adequate estimate of overland flow
for the land management applications (Flanagan and Nearing, 1995).
In addition, the flow is considered to be close to the
kinematic wave approximation
 for which ${\bf s_f}({\bf r},t) \approx {\bf s}({\bf r})$ and after
 using Equation (3), the Equation (1) is given by:

\leftequation
\nabla \cdot [ h({\bf r}){\bf v}({\bf r})] = i_e({\bf r})
%\eqno(5)
\endleftequation

\noindent
In order to incorporate the diffusive wave effects at least in an approximate way, 
Mitas and Mitasova (1998) incorporate a diffusion-like term
$ \propto \nabla^2 [h^{5/3}({\bf r})]$ into Equation (5):

\leftequation
-{\varepsilon({\bf r})\over 2 }\nabla^2 [h^{5/3}({\bf r})]
+\nabla \cdot [ h({\bf r}){\bf v}({\bf r})] = i_e({\bf r})
%\eqno (6)
\endleftequation

\noindent
 where $\varepsilon({\bf r})$ is a spatially variable diffusion coefficient.
Such an incorporation of diffusion in the water flow simulation is not new
 and a similar term has been obtained in derivations of diffusion-advection 
equations for overland flow, e.g.,  by Dingman, (1984) and Lettenmeier and Wood, (1992).
The diffusion term, which depends on $h^{5/3}({\bf r})$ instead of
 $h({\bf r})$, makes Equation (6)  {\sl linear}
 in the function $h^{5/3}({\bf r})$ which enables it to be 
 solved using the path sampling method.


\medskip
\noindent
\subsubsection{Erosion and sediment transport by overland flow}

\quad

\medskip
The basic relationship describing sediment transport by overland flow
is the sediment continuity equation,
which relates the change in sediment storage over time, and the change
in sediment flow rate along the hillslope to effective sources and sinks
(e.g., Haan et al., 1994; Govindaraju and Kavvas, 1991;
Foster and Meyer, 1972; Bennet, 1974).
The bivariate form of the continuity of sediment mass equation is
(e.g., Hong and Mostaghimi, 1995):
\leftequation
{\partial [\rho_sc({\bf r},t)h({\bf r},t)] \over \partial t} +
\nabla\cdot {\bf q_s}({\bf r},t) = {\rm sources - sinks}=
D({\bf r},t)
%\eqno(7)
\endleftequation

\noindent
where
${\bf q_s}({\bf r},t)$ in $kg/(ms)$ is the sediment flow rate per unit width,
$c({\bf r},t)$  in $ particle/m^3$ is sediment concentration,
$\rho_s$ in $kg/particle$ is mass per sediment particle,
$\rho_s c({\bf r},t)$ in $kg/m^3$ is sediment mass density,
and $D({\bf r},t)$ in $kg/(m^2s)$ is the net erosion or deposition rate.
The sediment flow rate ${\bf q_s}({\bf r},t) $ is a function of
water flow and sediment concentration:

\leftequation
  {\bf q_s}({\bf r},t) = \rho_sc({\bf r},t){\bf q}({\bf r},t)
%\eqno (8)
\endleftequation

\noindent
Again, we assume a steady state form of
%For shallow, gradually varied flow the storage term can be
%neglected leading to a steady state form of
the continuity equation:
\leftequation
{\partial [\rho_sc({\bf r},t)h({\bf r},t)] \over \partial t}=0
\quad
\longrightarrow \quad
\nabla\cdot {\bf q_s}({\bf r}) = D({\bf r}) .
%\eqno(9)
\endleftequation
\noindent
%where $D({\bf r})\; [kg/(m^2s)]$ is the net erosion/deposition rate.
The sources and sinks term is derived from the assumption that
the detachment and deposition rates are proportional to the difference between
the sediment transport capacity and the actual
sediment flow rate (Foster and Meyer, 1972):

\leftequation
D({\bf r})
=\sigma({\bf r}) \bigl[ T({\bf r})- |{\bf q_s}({\bf r})|\bigr]
%\eqno(10)
\endleftequation

\noindent
where $T({\bf r})$ in $kg/(ms)$ is the sediment transport capacity,
$\sigma({\bf r})$ in $m^{-1}$ is the first order reaction term
 dependent on soil and cover properties.
 The expression for $\sigma({\bf r})=D_c({\bf r})/T({\bf r})$
is obtained from the following relationship (Foster and Meyer, 1972):

\leftequation
D({\bf r})/D_c({\bf r}) + |{\bf q_s}({\bf r})|/T({\bf r}) = 1
%\eqno(11)
\endleftequation

\noindent
 which states that the ratio of the erosion rate to the detachment
 capacity $D_c ({\bf r})$ in $kg/(m^2s)$ %(entrainment)
 plus the ratio of the sediment flow to the sediment transport capacity is
 a conserved quantity (unity). To keep the model simple, $\sigma({\bf r})$
is applied for both net erosion and deposition.
%Adding a different equation for deposition 
%is necessary for accurate simulations of sedimentation in ponds,
%streams and other 3D water bodies 
%which are not currently included in the model and which
%will require simulation of 3D flow.
Equation (11), proposed by Foster and Meyer (1972),
is based on the observed relationship between soil detachment and transport
described e.g., by Meyer and Wischmeier
(1969). This concept is used in several erosion models including {\sl WEPP}
(Haan et al., 1994; Flanagan and Nearing, 1995).
The qualitative arguments, experimental observations and values for
 $\sigma({\bf r})$   are discussed by Foster and Meyer, (1972)
and Foster (1982). While it is possible to use other frameworks
for estimation of  $\sigma({\bf r})$, we have chosen the Foster and
Meyer concept because of its simplicity and widespread use.

The sediment transport capacity $T({\bf r})$ and detachment capacity
$D_c({\bf r})$ represent the maximum potential
sediment flow rate and the maximum potential detachment rate, respectively,
%by overland flow
and there are numerous simplified equations representing these rates under
different conditions. In the {\sl WEPP} model they are expressed as functions
of a shear stress (Foster and Meyer, 1972):

\leftequation
T({\bf r})=K_t({\bf r}) \bigl[\tau({\bf r})\bigr]^p
%=K_t({\bf r}) \bigl[\rho_w\, g h({\bf r}) \sin \beta ({\bf r}) \bigr]^p
%\eqno(12)
\endleftequation
\leftequation
D_c({\bf r})=K_d({\bf r}) \bigl[\tau({\bf r}) - \tau_{cr}({\bf r})\bigr]^q
%=K_d({\bf r}) \bigl[\rho_w\, g h({\bf r}) \sin \beta({\bf r}) -\tau_{cr}({\bf r})\bigr]^q
%\eqno (13)
\endleftequation

\smallskip
\noindent
where $\tau({\bf r})=\rho_w\, g h({\bf r}) \sin \beta({\bf r})$ in
$Pa$ is the shear stress,
$ \beta$ in $deg$ is the slope angle,
$p$ and $q$ are exponents,
$ K_t({\bf r})$ in $s$ is the effective transport capacity coefficient,
%($[s]$ for $p=1$),
$K_d({\bf r})$ in $s/m$ is the effective erodibility
(detachment capacity coefficient),
%($[s/m]$ for $p=1$),
$\rho_w\, g$  is the hydrostatic pressure
of water with the unit height,
$ g=9.81$ in $m/s^2$ is the gravitational acceleration,
$\rho_w=10^3$ in $kg/m^3$ is the  mass density of water,
and
$\tau_{cr}({\bf r})$ in $Pa$ is the critical shear stress.
The parameters and adjustment factors for the estimation of
$D_c ({\bf r})$ in $T({\bf r})$
are functions of soil and cover properties, and their values
for a wide range of soils, cover, agricultural and 
erosion prevention practices
were developed within the {\sl WEPP} model (Flanagan and Nearing, 1995).

The exponents $p, q$ substantially influence model behavior.
The {\sl WEPP} model uses $q=1$ and $p=1.5$ which means that with increasing
water flow, transport capacity increases faster than detachment.
The comparison with the spatial extent of colluvial deposits
in our previous study (Mitas and Mitasova, 1998)
indicates that over long term period (decades) 
the pattern of erosion/deposition
is closer to the results obtained with a lower value of $p$.
This is in agreement with several studies
which indicate that for the landscape scale modeling
the Equations (12) and (13) are not general enough for different
types of flow and sediment transport processes
present in complex landscapes
(Willgoose et al., 1989; Kirkby, 1987; Willgoose and
Gyasi-Agyei, 1995; Mitas and Mitasova, 1998).

%Nearing et al. (1997) proposed a transport capacity equation as a
%function of stream power. The equation was re-arranged (Mitas and
%Mitasova, 1997) and implemented within the presented model.
%Its advantage is that it
%better reflects the influence of change in flow velocity
%on the net erosion/deposition.
%However, for the practical use in land management the values of coefficients
%used in this equation should be calibrated for different soil and
%land cover conditions. 

Similar to the water flow equation, the steady
state sediment flow equation (Equation 9) can be rewritten to include a 
diffusion term. First, a function representing the mass of water-carried 
sediment per unit area $\varrho({\bf r})$ in $kg/m^2$ 
is defined as:

\leftequation
\varrho({\bf r})= \rho_s c({\bf r})h({\bf r})
%\eqno(14)
\endleftequation

\noindent
and after introducing a small diffusion term $\propto \nabla^2 \varrho({\bf r})$,
the continuity equation is rewritten as:
\leftequation
-{\omega \over 2}\nabla^2 \varrho({\bf r})
+ \nabla\cdot [\varrho({\bf r}){\bf v}({\bf r})]
 + \varrho({\bf r}) \sigma({\bf r}) |{\bf v}({\bf r})|
= \sigma({\bf r}) T({\bf r})
%\eqno(15)
\endleftequation

\noindent
where  $\omega$  in $m^2/s$ is the diffusion constant.
 On the left hand side of Equation (15) the first term
 describes local diffusion,
 the second term is a drift driven by the water flow
 while the third term represents a velocity dependent
'potential' acting on $\varrho({\bf r})$. The size of the diffusion
 constant is about one order of magnitude smaller than the
 reciprocal Manning's constant so that the impact of the diffusion term
is relatively small. It represents local dispersion processes
of the suspended flow (Bennet, 1974), caused by microtopography
which is not captured by the {\sl DEM}.  
The diffusion term can be modified to reflect impact of various processes.

The water and sediment flow described by Equations (6) and (15)
can be solved by the path sampling stochastic approach described in the following
section.

\bigskip \medskip \noindent
\subsection{Path Sampling  Solution  Method}

\medskip
Equations (6) and (15) have a similar form in which
 a linear differential operator $\cal O$ acts on a nonnegative
unknown function
 $\gamma({\bf r})$ (either $h({\bf r})$ or $\varrho ({\bf r})$),
  while on the right hand side, there is a
given source term ${\cal S}({\bf r})$ (either $i_e({\bf r})$ or
$\sigma({\bf r})T({\bf r})$):

\leftequation
{\cal O} \gamma({\bf r})= {\cal S}({\bf r})
%\eqno(16)
\endleftequation

\noindent
Denoting ${\cal O}^{-1}$ the inverse operator to ${\cal O}$,
 the solution can be symbolically written as:
\leftequation
\gamma({\bf r})=  {\cal O}^{-1} {\cal S}({\bf r})
%\eqno(17)
\endleftequation

\noindent
or explicitly, using the Green's function:
\leftequation
\gamma({\bf r})=
\int_0^{\infty}\int
G({\bf r},{\bf r'},p)
 {\cal S}({\bf r'})
d{\bf r'}dp
%\eqno(18)
\endleftequation

\noindent
The Green's functions are often used for expressing
the solutions of linear differential equations in
physical or mathematical applications. For the theory see, e.g.,
 Karlin and Taylor, (1981); Glimm and Jaffe, (1972);
Stakgold, (1979); Carslaw and Jaeger, (1947).
$G({\bf r},{\bf r'},p)$
is given by the following equation and an initial condition:

\leftequation
{\partial G({\bf r},{\bf r'},p)
\over \partial p} =-{\cal O} G({\bf r},{\bf r'},p)\, ;
\;\; G({\bf r},{\bf r'},0) =\delta ({\bf r}-{\bf r'})
%\eqno(19)
\endleftequation

\noindent
 where ${\bf r},{\bf r'}$ are locations,
$p$ is time and $\delta$ is the Dirac function. In addition,
 we assume that the spatial region
 is a delineated watershed with zero boundary condition which is
 fulfilled by $G({\bf r},{\bf r'},p)$. The corresponding equations
 can be solved, e.g., by projection methods (Rouhi and Wright, 1995).
 Another equivalent
 alternative is to interpret Equations (6),(15) and (16) as describing
 stochastic processes with diffusion and drift components
  (Fokker-Planck equations)
 and carry out the actual simulation of the underlying process
  using a path sampling method (Gardiner, 1985).

The method is based on duality between the particle and field 
representation of spatially distributed phenomena.Within this concept,
 density of particles in
space defines a field and vice versa, field is represented by
particles with corresponding spatial distribution of their densities. Using
this duality, processes can be modeled as evolution of fields or evolution 
of spatially distributed particles (Figure 1), 
with the solution obtained as follows.  

\begin{figure}[h]
\centerline{\epsfig{file=llefig1.eps,width=1.\textwidth,clip= }}
\caption{Path sampling solution of the continuity equation for water 
depth $h({\bf r})$ using duality between particle and field representation:
a) water depth at 1 minute, b) water depth after 24 minutes. The grid
is 416x430 cells at 10m resolution. See the {\sl CDROM} for animation.}
\end{figure}

 First a selected number of particles, also called walkers or sampling points,
is distributed according to the source 
$ {\cal S}({\bf r'})$. These walkers are then propagated
 according to the function $G({\bf r},{\bf r'},p)$, generating a number
of sampling paths. Averaging of these path samples provides
 an estimation of the actual solution $ \gamma ({\bf r})$ 
 with statistical accuracy proportional to $1/\sqrt{M}$ where $M$ 
is the number of walkers (Figure 2). The solution is not restricted to the 
steady state and the state of the modeled quantity at any given time
$p$ can be obtained by averaging the path samples at a given time $p$.   

\begin{figure}[h]
\centerline{ \epsfig{file=llefig2.eps,width=1.\textwidth,clip= } }
\caption{Path sampling solution of the continuity equation for sediment
flow and net erosion/deposition:
a) results for 7000 walkers, b) results for 50 million walkers.
 The grid is 280x250(70,000 cells) at 2m resolution. See the {\sl CDROM} 
for animation.}
\medskip
\end{figure}

The path sampling technique has several unique advantages which
are becoming even more important due to new developments in
computer technology. Perhaps one of its most significant
 properties is robustness which makes it possible to solve the equations
 for complex cases, such as discontinuities in
 the coefficients of differential operators (in our case, abrupt
 slope or cover changes, etc). In addition, the independence of sampling points
makes the stochastic methods perfectly suited to
the new generation of computers as they provide scalability from
a single workstation to large parallel
 machines and computers distributed over different types of networks.

\bigskip
\noindent
\section{SIMPLIFIED SPECIAL CASES AND MODEL EXTENSIONS}

Land use management poses specific challenges for hydrologic and erosion
simulations because of the necessity to capture the spatial aspects
of the modeled processes.
In the following sections some of the issues
important from the point of view of landuse design
are addressed, in particular the
simplified erosion and deposition models,
simulation of water flow in flat areas and depressions,
and modeling with spatially variable resolution.

\bigskip \medskip \noindent
\subsection{Simple Erosion and Deposition Models }

\medskip
To satisfy the need for models which are easy to implement, 
simple to compute, and for which the data a readily available,
it is useful to derive the models  for special cases
of sediment transport regimes.
There are two limiting cases of erosion and sediment transport
(Foster and Meyer, 1972; Hairsine and Rose, 1992; Tucker et al., this volume):
(i) {\sl detachment limited},
and (ii) {\sl sediment transport capacity limited}.

\medskip
\noindent
\subsubsection{The detachment limited case} 

\quad 

\medskip
This case is represented by $\sigma \to 0$, which,
after substituting into Equations (9)-(10) results in the net erosion
equal to the detachment capacity 
$ D({\bf r})=D_c({\bf r})$.
%= K_d({\bf r}) \bigl[\rho_w\, g h({\bf r}) \sin \beta({\bf r})
%-\tau_{cr}({\bf r})\bigr]^q
It can be directly computed from Equation (13) using an estimate of
water depth $ h({\bf r})$ computed, for example, by the {\sl SIMWE}
 water flow module. 
%or  using the upslope contributing area per unit width
%$A({\bf r})$ as $h({\bf r}) = 
%[n({\bf r}).A({\bf r}).i_e({\bf r})]^0.6/[\sin \beta({\bf r})]^0.3$.
%Monthly/annual average detachment limited erosion can 
%then be computed as a summ of $D({\bf r})$ for all events in a month/year,
% averaged over a number of years.
 For this case, transport capacity exceeds
detachment capacity everywhere, erosion and sediment transport is detachment
capacity limited and therefore no deposition occurs.

The most common erosion model which represents this case is
{\sl USLE} and its revised version {\sl RUSLE} (Lane et al., this volume).
If we assume that $\tau_{cr}({\bf r})=0$
and that the spatial distribution of steady state water flow 
is adequately represented by a function of
upslope contributing area per unit width (e.g., Moore et al., 1993) 
the detachment limited erosion can be approximated by
{\sl RUSLE3D}, which for a point on a hillslope
has the form (Moore et al., 1993; Mitasova et al., 1999; 
see also Desmet and Govers, 1996 for hillslope segment based equation):
\smallskip
\leftequation
D({\bf r})=R({\bf r}) K({\bf r}) C({\bf r}) P({\bf r})
(m+1) [A({\bf r}) / 22.13]^m [sin \beta({\bf r}) / 0.09]^n
%\eqno(20)
\endleftequation
\smallskip
\noindent
where
$D({\bf r})$ in $ton/(acre.year)=0.2242kg/(m^2.year)$ 
 is the average annual soil detachment (soil loss) rate,
$A({\bf r})$ in $m^2/m$ is the upslope contributing area per unit width,
$22.13m$ is the length and $0.09$ is the slope of the standard
USLE plot, $R({\bf r})$ in $(hundreds\,of\,ft-tonf.in)/(acre.hr.year)$
$=17.02 \; MJ.mm/(ha.hr.year)$ is the rainfall energy factor, 
$K({\bf r})$ in
$ton.acre.hr/$$(hundreds\,of\,acre ft-tonf.in)=0.1317\,
t.ha.hr/$$(ha.MJ.mm)$ is
soil erodibility, $C({\bf r})$ [dimensionless] is the cover factor and
$P({\bf r})$ [dimensionless] is the prevention measures factor 
(Haan et al., 1994; Lane et al., this volume). Single storm and
monthly $R({\bf r})$ is also available, making Equation (20) suitable for
estimation of $D({\bf r})$ for single storms and for modeling of
monthly soil loss distribution over a year (Haan et al., 1994). 
Exponents $m,n$ depend on the prevailing type of
erosion (sheet, rill) and the typical values are $m=0.4-0.6$ and $n=1-1.3$.
Replacement of slope length used in the original formulation of 
{\sl USLE/RUSLE} by the upslope area provides a better spatial description
of increased erosion due to the concentrated flow without the need
to {\sl a priori} define these locations as inputs for the model.

\medskip
\noindent
\subsubsection {Transport capacity limited erosion} 

\quad

\medskip
For this case $\sigma \to \infty$,
which, after substituting into Equation (10), leads to 
$|{\bf q}_s({\bf r})| \approx T({\bf r})$ and
net erosion/deposition can be computed directly as a divergence of
the sediment transport capacity:
\leftequation
D({\bf r})=\nabla\cdot {\bf q}_s({\bf r})=
\nabla\cdot \bigl[T({\bf r}){\bf s_0}({\bf r})\bigr]
%\eqno(21)
\endleftequation

\noindent where
${\bf s_0}({\bf r})={\bf s}({\bf r})/|{\bf s}({\bf r})|$ is the
unit vector in the steepest slope direction.
Mitas and Mitasova (1998) have shown that the results obtained for
this case were close to the observed distribution of colluvial deposits
in their study area, suggesting the prevailing influence
of the transport capacity limited case on a long term
pattern of deposition. Equation (21) was also used to demonstrate
the impact of both tangential and profile curvatures 
and the importance of 2D flow routing
for predicting net erosion/deposition pattern (Mitas and Mitasova, 1998).  

The Unit Stream Power Based Erosion/Deposition model ({\sl USPED})
estimates the transport capacity limited case of erosion/deposition
using the idea originally proposed by Moore and Burch (1986).
It combines the {\sl USLE/RUSLE} parameters and 
 upslope contributing area per unit width $A({\bf r})$
to estimate the sediment flow at sediment transport capacity: 
\leftequation
q_s({\bf r}) = T({\bf r}) \approx R({\bf r}) K({\bf r})C({\bf r}) P({\bf r})
[A({\bf r})]^m [sin \beta({\bf r})]^n
%\eqno (22)
\endleftequation

\noindent
%ROZVED?The upslope area and slope are not normalized, because Equation (22) is
%here an estimate of sediment flow $[kg/(ms)]$ rather than soil detachment
%$[kg/(m^2s)$ (see the discussion about the use of RUSLE parameters below). 
The net erosion/deposition $D({\bf r})$ is then estimated using Equation (21) as:

\leftequation
D({\bf r})=\nabla\cdot \bigl[T({\bf r}){\bf s_0}({\bf r})\bigr]
= {d [T({\bf r}) cos \alpha({\bf r})] \over {dx}} +
{d [T({\bf r}) sin \alpha({\bf r})) \over {dy}}
%\eqno (23)
\endleftequation

\noindent
where $\alpha({\bf r})$ in $deg$
is the aspect of the terrain surface (direction of flow).
The exponents $m,n$
control the relative influence of water and slope terms and reflect
the impact of different types of flow.
The exponents leading to values of erosion rates on hillslopes
consistent with the values from {\sl RUSLE} are $m=1.6, n=1.3$
and they seem to reflect the pattern for prevailing rill erosion
with erosion sharply increasing with the amount of water. The observed
extent of colluvial deposits in our previous study (Mitas and Mitasova, 1998)
 indicated that a lower exponent $m=1$ better reflects the pattern of compounded,
long term impact of both rill and sheet erosion and averaging over a long
term sequence of large and small events.

Models representing limiting cases of erosion are simple to compute
in GIS by combining the flow-tracing and topographic analysis functions with
map algebra. They can be applied to a single storm, monthly
and annual estimates of soil detachment and net erosion/deposition.
 Caution should be used when interpreting the results from both 
{\sl RUSLE3D and USPED} because the {\sl USLE/RUSLE}
parameters were developed for simple plane fields and detachment limited
erosion. Therefore, to obtain accurate quantitative predictions for
complex terrain and land cover conditions
they need to be re-calibrated, especially in areas of concentrated flow
 (Foster, 1990; Mitasova et al., 1997 reply).
While the capabilities of both {\sl RUSLE3D and USPED}
to accurately predict the rates of
erosion and deposition at any point in the complex landscape are limited (fact
which is true about almost any erosion model),
they are useful tools for land management.
Both models use readily available parameters and
can provide valuable spatial information about: (i) the location of areas
with high erosion risk from both shallow overland and concentrated flow,
(ii) location of areas with deposition, and
(iii) relative estimates of erosion and deposition rates
for different land use alternatives and conservation strategies.
Locations identified as high risk from both {\sl RUSLE3D and USPED}
should be primary targets for field erosion inventory
(to validate the risk) and implementation of prevention/mitigation measures
(if the high risk is confirmed in the field). Computation of net erosion
and deposition is also useful for evaluation of the landscape's capacity
to deposit the eroded material before it can reach the streams.
%Combined application of {\sl USLE3D and USPED} provides information about the
%erosion patterns for limiting cases of erosion regimes
%It is interesting to note that
%prevailing long term pattern indicates transport capacity limited case -
%{\bf trimble wisconsin !!!!!.....} Address the concentrated flow uncertainty

\bigskip \medskip \noindent 
\subsection{Water Depth in Flat Areas and Depressions}

\medskip
While flat areas and depressions are not high erosion risk locations,
they play an important role within watersheds by holding water
and reducing water flow to neighboring steeper slopes or into the streams.
Modeling water flow over terrain surfaces with $\nabla z>0$ can be
successfully performed using kinematic wave approximation. However,
flat areas and bottoms of depressions where $\nabla z \to 0$
pose a problem because the water flow direction becomes undefined.
Incorporation of the spatially variable diffusion term $\varepsilon({\bf r})$
in the SIMWE model provides capabilities to approximately simulate water depth
in these locations. By defining the diffusion term $\varepsilon({\bf r})$
as a function of water depth and the 

\begin{figure}[h]
\centerline{\epsfig{file=llefig3.eps,width=1.\textwidth,clip= }}
\caption{Simulation of water flow through existing shallow depressions:
a) 2D kinematic wave flow, b) approximate 2D diffusive wave flow
with variable $\varepsilon({\bf r})$ and flow momentum,
c) approximate 2D diffusive wave flow with a shallow, 1 degree slope
channel running through the depressions. Simulation was performed
by the {\sl SIMWE} model using a 10m resolution {\sl DEM}, 100x170 cells.
See the {\sl CDROM} for animation.}
\end{figure}

\noindent
velocity of flow as a function of an approximate water flow momentum,
the  water fills the depressions or spreads in the flat area
and flows out in the prevailing flow direction (Figure 3b).

For the situations when the flat areas or depressions are drained by natural
or man-made swales and channels, the water flow
can be simulated by combining the gradient field derived from the {\sl DEM} with
 the gradient of the drainage (Figure 3c). The input vector field
${\bf s(r)}$ representing
the flow direction is then defined as:

\leftequation 
{\bf s}({\bf r})=
(1-\delta_{ij}){\bf s_e}({\bf r}) + \delta_{ij} {\bf s_d}({\bf r})
%\eqno (24)
\endleftequation

\noindent
where ${\bf s_e(r)}$ is the vector in the steepest slope direction derived
from a {\sl DEM} and ${\bf s_d(r)}$ is the vector representing the
flow through the surface drainage which can be estimated, for example, 
from drainage line data using GIS tools and/or from field measurements.

\bigskip \medskip \noindent
\subsection{Multiscale Water and Sediment Flow Simulation}

\medskip
Because the spatial unit modeled at the site
level is part of a larger watershed, the evaluation of the impact of
numerous, locally implemented conservation practices across the entire watershed
requires multiscale approach. This approach links the high resolution, local level
simulation with low resolution/regional simulation and is being implemented
for the {\sl SIMWE} model. It supports simulations with spatially
variable resolution which can include the following cases:
(i) study area is represented by several data sets with
different resolutions and levels of accuracy, and
the best available data are used for each subregion;
(ii) study area is large, with spatially variable complexity
and it is sufficient to run the more homogeneous areas at lower resolution
while running the more complex areas or areas experiencing land use
change at high resolution.

\noindent
Both {\it spatially variable accuracy} and {\it resolution} can be
implemented by reformulating the  solution through the Green's function
given by Equation (18).
The integral Equation (18) can be multiplied by a {\it reweighting }
 function  $W({\bf r})$:

\leftequation
W({\bf r})\gamma({\bf r})= \int_0^{\infty}\int W({\bf r})G({\bf r},{\bf r'},p)
 {\cal S}({\bf r'})  d{\bf r'}dp =
\endleftequation
\leftequation
= \int_0^{\infty}\int G^*({\bf r},{\bf r'},p)
 {\cal S}({\bf r'})  d{\bf r'}dp
%\eqno(25)
\endleftequation

\noindent
 which is equal to the appropriate increase in accuracy ($W({\bf r})>1$) in the 

\begin{figure}[h]
\centerline{\epsfig{file=llefig4.eps,width=1.\textwidth,clip= }}
\caption{Simulation of water depth at spatially variable
resolution 10m and 2m: a) initial particle representation,
 b) detail of resulting water depth at 10m resolution
c) detail at 2m resolution. See the {\sl CDROM} for animation} 
\end{figure}

\noindent
%The low and high resolution areas are 416x430 and 660x375 respectively.
regions of interest while it is unity elsewhere. The function $W({\bf r})$ can 
change (abruptly or smoothly) between regions with unequal
 resolutions  and in fact, can be optimally adapted to the quality of
 input data (terrain, soils, etc) so that the more accurate solution
 is calculated only in the regions with correspondingly accurate inputs.

 The reweighted Green's function $G^*({\bf r},{\bf r'},p)$, in effect,
  introduces a higher density of sampling points in the region with
 large $W({\bf r})$. The statistical noise will be spatially
 variable as $\approx 1/[W({\bf r})\sqrt{M}]$, where $M$ is the average number
 of samples  resulting in the accuracy increase for the areas
with $W({\bf r}) > 1$.

This approach provides an alternative to
the finite element methods (see e.g. Tucker et al., this volume)
 because it uses
multiple standard 
grids with the given resolutions
instead of finite element meshes, which often
 lack adequate {\sl GIS} support.
The implementation is based on the multipass simulation.
First, the entire area is simulated at lower resolution, and the walkers
entering the high resolution area(s) are saved.
The saved walkers are resampled according to the Equation (25) by
spliting each walker into a number of "smaller" walkers which are
randomly distributed in the neighborhood of the original walker.
The model is then run at high resolution only for the given subarea,
with the resampled walkers used as inputs (Figure 4). 
If several different land use alternatives are considered
for the given subarea, this approach can be used to perform
simulations for each alternative only within the high resolution subarea.
The approach also provides useful spatial information
about the locations where water flows into the given subarea
and where it flows out (Figure 4c).


\bigskip \noindent
\section{LANDSCAPE SCALE EROSION PREVENTION PLANNING AND DESIGN}

Interactions between different land covers and topography
significantly influence the spatial distribution of surface water depth,
sediment flow and net erosion/deposition. The capabilities to
simulate these interactions at both watershed and field scale
can support the design of sustainable, cost effective
conservation strategies and erosion prevention measures.
Mitasova and Mitas (1998) have demonstrated the use of {\sl SIMWE}
and {\sl GIS} for finding an effective spatial distribution of
protective grass cover for a small agricultural watershed.
The following sections provide examples of
a wide range of applications where simulations of water flow,
sediment transport and erosion/deposition are used
to support land use management at different scales
and levels of complexity.

\bigskip \medskip \noindent
\subsection{ Watershed Scale Erosion Risk Assesment and Evaluation of 
Conservation Strategies with Simple Distributed Models}

\medskip
A large number of watershed associations are being organized 
({\sl EPA}, 2000)
with the goal of improving the management of
America's watersheds. The basis for this work are
watershed management plans which identify the problems and
set priorities in funding and implementation of conservation measures.
{\sl GIS} combined with simple erosion models and free spatial data
available through the National Spatial Data Clearinghouse ({\sl NSDI}, 2000)
provide a cost effective way to assess
the current state of watersheds, as well as evaluate the impact and
prioritize various conservation strategies.
The simplified models {\sl RUSLE3D and USPED}
were applied to the Court Creek watershed which
serves as a pilot area for the Illinois Department of Natural
Resources ({\sl IDNR}) program aimed at demonstration of
community based watershed management with strong scientific support.
First, the current erosion risk areas were identified and their pattern
was analyzed using {\sl RUSLE3D} and {\sl GIS} tools (Figure 5b).
The analysis shows a favorable land use pattern
with protective forested buffers along the bigger streams and
on steep slopes. However, the analysis also indicates that
headwater areas and areas with lower values of slope and convergent water
flow are not sufficiently protected. These sensitive areas are relatively
small and scattered, and the results from {\sl RUSLE3D}
indicate that only 16 percent area (10,000 acres) produces
87 percent of total detached soil available for transport.

The impact of several conservation strategies was then evaluated,
with the following two alternatives presented here:
(i) 30m protective buffers along the bigger streams with rest of the
watershed in agriculture;
(ii) critical area planting of conservation areas based on the erosion risk map.
The comparison of the strategies in terms of gain or loss of
agricultural land and reduction of erosion for a high risk Court Creek 
subwatershed is presented in Figure 5. The analysis demonstrates that a 30m
buffer along the main stream does not provide adequate erosion
protection. While it would make 94 percent of the area available to
agriculture, it would also lead to a three fold increase in 
average annual soil loss.
Elimination of high erosion potential would require reduction
of agricultural land by only 5 percent and extension of the criteria
currently used for the conservation program by including the
headwater areas and areas with convergent water flow.

The results obtained from the {\sl USPED} model 
indicate that a substantial portion of the eroded soil moves
only for a short distance and there is enough concave areas
to deposit the sediment before it can enter the 

\begin{figure}[h]
\centerline{\epsfig{file=llefig5.eps,width=1.\textwidth,clip= }}
\caption{Court Creek subwatershed soil detachment
for different land use alternatives estimated by {\sl RUSLE3D}:
a) 30m buffers along the streams, 94 percent row crops, soil loss 27t/(ac.yr),
b) current land use, 63 percent row crops/grains, soil loss 8t/(ac.yr),
c) grass cover in areas with soil detachment>10 t/(acre.yr), 58 percent
 row crops, soil loss 1 t/(acre.yr).
The area is 3.6x4.6km modeled at 10m resolution.
(see also http://www2.gis.uiuc.edu:2280/modviz/courtcreek/cc.html)}
\end{figure}

\noindent
streams.

These results support some recent observations and
hypotheses (Roseboom and Mollahan, 1999; Trimble, 1999)
that in the Midwestern watersheds most of the sediment observed in the streams
originates within the streams and from erosion by concentrated flow
rather than from hillslope erosion by shallow overland flow.

This application demonstrates that the simple models used with
widely available data can be useful for preliminary assessment
of erosion and sedimentation risk, identification of "hot spots"
in the watersheds and approximate evaluation of different
conservation strategies.
%Design and implementation of best management practices with reduced maintenance cost
%and increased effectiveness requires high resolution data
%and process based modeling.
%Application of {\sl SIMWE} to different land use design
%tasks illustrates the type of problems that the simulations
%can help to understand and solve.

\bigskip \medskip \noindent
\subsection{Wetlands and Drainage}

\medskip
\noindent
\subsubsection {Topographic potential for wetlands}

\quad

\medskip
Preservation and restoration of wetlands is among the most
important and popular best management practices. Their success
depends on many factors, including
a sufficient supply of water.
The {\sl SIMWE} hydrologic submodel was used to identify
the locations  within the Court Creek Pilot Watershed which have
topographic conditions favorable for 
\begin{figure}[h]
\centerline{\epsfig{file=llefig6.eps,width=1.\textwidth,clip= }}
\caption{ Map of topographic potential for wetlands.
Existing wetlands are displayed as polygons and cover 1\% of the subwatershed
while the model identifies 6\% area as suitable for
wetlands. The area is 4x6km, simulated at 10m
resolution (resampled from 30m resolution {\sl DEM})}.
\end{figure}

\noindent
wetlands.
Several simulations were performed for various rainfall intensities,
uniform land cover and saturated soil conditions, assuming that the flow
velocity is controlled only by the terrain gradient - the existing drainage
and channels were not considered.
Comparison of the simulated water depth with existing
wetland areas shows that these areas are characterized by 
steady state water depth from one event of
at least 0.3m. Using this threshold,
a map for topographic potential for wetlands was
computed using map algebra (Figure 6). While the simulation was very simplified,
the map can serve as a useful starting point for
identification of land owners with suitable land for wetlands
and for evaluation of the proposals for wetland locations.


\medskip\noindent
\subsubsection{Drainage location design}

\quad

\medskip
Simulation of spatial distribution of water depth provides valuable
information also for an "opposite" task - identification of locations which
require drainage to prevent negative impact of standing water on yields.
Using a high accuracy {\sl DEM}
( 6m resolution, 0.05m vertical accuracy) interpolated from
rapid kinematic survey data by the {\sl RST} method (Mitas and Mitasova, 1999),
the water depth distribution was simulated for a typical
rainfall for Midwestern agricultural fields (9mm/hr)
under saturated conditions. The resulting water depth map
was used to evaluate suitability of the locations of current drainage
and to plan the location
of new drainage network in the negatively affected field (Figure 7).
While the model was very useful for evaluating and planning
a suitable spatial pattern of the drainage network,
detailed soil data and
more complex dynamic simulations are needed to design the size, depth
and other parameters of the drainage.

\begin{figure}[h]
\centerline{\epsfig{file=llefig7.eps,width=1.\textwidth,clip= }}
\caption{ Simulated spatial distribution of water depth
for agricultural fields (2.5x4.5km) draped
as color over the {\sl DEM} (6m resolution,
30-times vertical exaggeration), with the existing drainage drawn as lines.}
\end{figure}

\bigskip \medskip \noindent
\subsection{Concentrated Flow Erosion and Grassed Waterways}

\medskip
The suitability of the {\sl SIMWE} model
for spatial design of vegetation based best management practices
was evaluated by application to small experimental watersheds
with planned or installed erosion prevention measures.

\medskip\noindent
\subsubsection{Concentrated flow erosion} 

\quad

\medskip
Development of high erosion
in areas of concentrated flow was studied
by performing simulations of water flow and net erosion deposition
for an experimental field with uniform land cover
(350x270m, modeled at 2m resolution; Zhang, 1999).
For a short rainfall event ending before the flow has reached
steady state, the maximum erosion rate was on the
upper convex part of the hillslope and there was only
deposition in the center of the valley (Figure 8a).
As the duration of the rainfall increased,
water depth in the center of the valley has grown
rapidly until it reached
a threshold when linear features with very high erosion rates
 developed within the depositional area,
indicating potential for gully formation (Figure 8b).
This effect is modeled by both USPED (Mitasova et al., 1996, 1999)
and SIMWE (Mitas and Mitasova, 1998), however, a smooth, high resolution 
DEM without artifacts is needed to realistically capture
this commonly observed phenomenon (see Figure 2c in Mitas and Mitasova, 1999).
% Increase in the roughness (Mannings
%n) in the field "delays" the onset of erosion in the center
%of the valley by making the water flow ridge wider and lower????
%(for n=0.01 the erosion occurs for XX rainfall intensity,
%for n=0.1 the erosion in the valley starts for XX rainfall)
This example also demonstrates that for a dynamic event modeling
incorporation of re-entrainment process is important and should be
incorporated into the {\sl SIMWE} model (Hairsine and Rose, 1992).

\begin{figure}[h]
\centerline{\epsfig{file=llefig8.eps,width=1.\textwidth,clip= }}
\caption{  Water depth and net erosion/deposition pattern for
18mm/hr rainfall excess for a) short event, with only deposition
in the valley center, b) long event leading to steady state flow, with
both high erosion and deposition in the valley center,
indicating a potential for gully formation.
The 350x270m field is modeled at 2m resolution. See animation 
on {\sl CDROM}.}
\end{figure}

\medskip\noindent
\subsubsection {Grassed waterways} 

\quad

\medskip
The common practice for prevention of erosion by
concentrated flow are grassed waterways. Their design is guided by the topographic
conditions and roughness within the grassed area, represented
by Mannings coefficient ({\sl SCS}, 1988). To investigate the impact
of a grassed waterway, the water and sediment flow as well as net
erosion/deposition pattern were simulated for a field within
the Scheyern experimental farm (Auerswald et al., 1996; Mitas and Mitasova, 1998)
for the bare soil conditions and after the installation
of grassed waterway with different values of roughness in the field.
For the bare field, there is a potential for gully formation
 (Figure 9a). After the installation
of grassed waterway the center of the valley
becomes a depositional area. However, if the roughness in the
field is several times smaller than in the grassed area, high
erosion develops around the waterway, potentially replacing one big
gully with two smaller ones. This "double
channeling" problem can substantially increase the cost of the waterway
 maintenance (Figure 9b). Increasing the roughness in the field
reduces the risk of double channeling and the transition
from erosion in the field to deposition in the grassed area is
relatively smooth (Figure 9c). An alternative solution combines contour
filter strip on the upper convex part of the hillslope with grassed
waterway (Mitas and Mitasova, 1998).

\begin{figure}[h]
\centerline{\epsfig{file=llefig9.eps,width=1.\textwidth,clip= }}
\caption{Impact of grassed waterway and differences in roughness
on sediment flow: a) bare field with gully potential in the center,
b) grassed waterway (light grey, n=0.1) and the bare field ( dark grey, n=0.01) 
with sediment flow along the grassed waterway (double channeling), 
c) grassed waterway (n=0.1) and the field with increased roughness
(n=0.05) without increase in sediment flow along the waterway
and smooth transition from erosion to deposition. See erosion/deposition 
in color on {\sl CDROM}.}
\end{figure}


\bigskip \noindent
\section{CONCLUSIONS}

This chapter is focused on methodology and applications of simulation
methods for prediction and solution of
 land management problems related to overland flow erosion.
The presented approach aims at
keeping the models, and particularly the number of required
input parameters, as simple as possible
while capturing the effects important for sustainable
land use design.
The applications demonstrate the need for a set of modeling
tools with different levels of complexity to support land use management
from strategic planning to design and implementation.
To satisfy this need three interrelated models were presented.

The first model {\sl SIMWE} is based on generalization of
hillslope erosion model used in {\sl WEPP} (Flanagan and Nearing, 1995).
It models erosion regimes from detachment to transport capacity limiting 
cases, includes approximate diffusive wave effect and  
it supports multiscale modeling which can be further
extended to incorporate multiscale-multiprocess simulations. 

{\sl RUSLE3D/USPED} differ from {\sl SIMWE} in that they model only the limited 
cases of erosion and sediment transport, however they use readily available
parameters and therefore are easy to implement and use
(see on-line tutorials for {\sl GRASS5} and {\sl ArcView} by 
Mitasova and Mitas, 1999a,b).
All of the presented models can be
used for single storms as well as for long term averages.

The applications of spatially continuous simulations
revealed gaps in the theory of erosion processes in complex landscapes,
especially in the mathematical description of
transport capacity suitable for complex landscapes.
Spatially distributed field experiments based on new technologies for field
data collection, monitoring and remote sensing
closely coupled with modeling are needed to improve our understanding
of complex interactions involved in erosion processes and bring the quantitative
accuracy of predictions (which is currently at about 50-150\%) to acceptable
and useful levels.

The report "New Strategies for
America's Watersheds" (Committee on Watershed Management, National Research
Council 1999) identifies simulation modeling
as one area of special promise for watershed management.
At the same time,
this report analyzes the current status in watershed modeling for decision
making and concludes that the available models and methods are outdated
and "a major modeling effort is needed to develop and implement state-of-the-art
models for watershed evaluations"(pp.160-161).
The presented approach along with other models presented in several 
chapters in this book,
are a step towards the development and implementation of such tools.


\bigskip
\noindent
{\bf ACKNOWLEDGMENTS}

\medskip
We would like to acknowledge the long term support for this research
from Geographic Modeling Systems Laboratory
director Douglas M. Johnston as well as {\sl GIS} assistance by William
M. Brown. The funding was provided by the {\sl USArmy CERL},
Strategic Environmental Research and Development Program - {\sl SERDP},
Illinois Council on Food and Agricultural Research - {\sl CFAR}
 and Illinois Department of Natural Resources.
We greatly appreciate the sharing of data by K. Auerswald, S. Warren,
K. Drackett, Zhang Yusheng, and D. Timlin.
Our special thanks goes to two reviewers for their thorough review of the paper,
as well as for their stimulating questions and comments which 
significantly helped to improve the revised version of this chapter.



%or 
%\begin{chapthebibliography}{<widest bib entry>}
%\bibitem[optional]{symbolic name}
%Text of bib item...
%\end{chapthebibliography}


%%%%%%%
% \kluwerbib will produce this kind of bibliography entry:
%
% Anderson, Terry L.,...
%   More bib entry here...
%
% \cite{xxx} will print without brackets around the citation.
%
% \bibitem with square bracket argument should be used, i.e.,
%    \bibitem[Anderson, 1999]{xxx}Bibliography entry...
%  Whatever is in square brackets will be printed when you use
%  \cite

\bigskip
\begin{chapthebibliography}{1}

\bibitem{}
{\ninerm
Auerswald, K., A. Eicher, J. Filser, A. Kammerer, M. Kainz, R. Rackwitz, 
J. Schulein, H. Wommer, S. Weigland, and K. Weinfurtner, 1996,
 Development and 
implementation of soil conservation strategies for sustainable land use 
- the Scheyern project of the FAM, in: {\it Development and Implementation of 
Soil Conservation Strategies for Sustainable Land Use}, 
edited by H. Stanjek, Int. Congress of ESSC, Tour Guide, II, pp. 25-68, 
Technische Universitaet Muenchen, Freising-Weihenstephan, Germany.

\bibitem{}
Bennet, J. P., 1974, Concepts of Mathematical Modeling of Sediment Yield, 
{\it Water Resources Research}, 10, 485-496. 

\bibitem{} Carslaw, H. S., and J.C. Jaeger, 1947, {\it Conduction of Heat in Solids},
Oxford University, London.


\bibitem{}
Desmet, P. J. J., and G. Govers, 1996, A GIS procedure for automatically 
calculating the USLE LS factor on topographically complex landscape units, 
{\it J. Soil and Water Cons.}, 51(5), 427-433.

%\bibitem{}
%Desmet, P. J. J., and G. Govers, GIS-based simulation of erosion 
%and deposition patterns in an agricultural landscape: a comparison 
%of model results with soil map information, {\it Catena}, 25, 389-401, 1995.

\bibitem{}
Dingman, S. L., 1984, {\it Fluvial hydrology}, Freeman, New York.

\bibitem{}
Doe, W.W., B. Saghafian, and P.Y. Julien, 1996, 
Land Use Impact on Watershed Response: 
The Integration of Two-dimensional Hydrological 
Modeling and Geographical Information Systems. 
{\it Hydrological Processes}, 10, 1503-1511.

\bibitem{}
EPA, 2000, Surf your watershed; http://www.epa.org.surf2/

\bibitem{}
Flanagan, D. C., and M. A. Nearing (eds.), 1995,
USDA-Water Erosion Prediction Project, 
{\it NSERL}, report no. 10, pp. 1.1- A.1, National Soil Erosion Lab., 
USDA ARS, Laffayette, IN.

\bibitem{}
Foster, G. R., 1982, 
Modeling the erosion processes, in: {\it Hydrologic modeling of small watersheds},
edited by C. T. Haan, H. D. Johnson, and D. L. Brakensiek, ASAE Monograph No. 5,
ASAE, St. Joseph, MI, pp. 197-380.

\bibitem{}
Foster, G. R., and L. D. Meyer, 1972, A closed-form erosion equation 
for upland areas, in: {\it Sedimentation: Symposium to Honor Prof. H.A. Einstein}, 
edited by H. W. Shen, pp. 12.1-12.19, Colorado State University, Ft. Collins, CO.

\bibitem{}
Foster, G. R., 1990, Process-based modelling of soil erosion by water on 
agricultural land, in: {\it Soil Erosion on Agricultural Land}, 
edited by J. Boardman, I. D. L. Foster and J. A. Dearing, 
John Wiley and Sons Ltd, pp. 429-445.

%\bibitem{}
%Foster, G. R., and W. H. Wischmeier, 1974) Evaluating irregular 
%slopes for soil loss prediction, {\it Trans. of ASAE}, 17, 305-309.

\bibitem{}
Gardiner, C. W., 1985,  {\it Handbook of Stochastic Methods for Physics, 
Chemistry, and the Natural Sciences}, Springer, Berlin.

\bibitem{} 
Glimm J., and A. Jaffe, 1972, {\it Quantum Physics. A Functional Integral Point
 of View}, Springer, Berlin.

\bibitem{}
Govindaraju, R. S., and M. L. Kavvas, 1991, Modeling the erosion process 
over steep slopes: approximate analytical solutions, 
{\it Journal of Hydrology}, 127, 279-305.

%\bibitem{}
%Govers, G., 1991)
% Rill erosion on arable land in central Belgium: rates, controls and predictability, {\it Catena}, 18, 133-155.

\bibitem{}
Haan, C. T., B. J. Barfield, and J. C. Hayes, 1994,
{\it Design Hydrology and Sedimentology for Small Catchments}, 
pp. 242-243, Academic Press.

\bibitem{}
Hairsine,  P. B., and C. W. Rose, 1992,  Modeling water erosion due to overland 
flow using physical principles 1. Sheet flow, 
{\it Water Resources Research}, 28(1), 237-243.

\bibitem{}
Hong, S., and S. Mostaghimi, 1995, Evaluation of selected management practices 
for nonpoint source pollution control using a two-dimensional simulation model, 
{\it ASAE}, paper no. 952700. Summer meeting of the ASAE, Chicago, IL.

\bibitem{}
Johnston, D.M. and A. Srivastava, 1999, Decision Support Systems for Design 
and Planning: The Development of HydroPEDDS (Hydrologic Performance 
Evaluation and Design Decision Support) System for Urban Watershed Planning, 
in: {\sl 6th International Conference on Computers in Urban Planning 
and Urban Management (CUPUMS'99)}, Venice, Italy (CDROM).

%\bibitem{}
%Julien, P. Y., and D. B. Simons, 1985)
%Sediment transport capacity of overland flow, {\it Transactions of the ASAE}, 28, 755-762, 1985.

\bibitem{}
Julien, P. Y., B. Saghafian, and F. L. Ogden, 1995, Raster-based hydrologic modeling of spatially
varied surface runoff, {\it Water Resources Bulletin}, 31(3), 523-536.


\bibitem{} Karlin, S., and H. M.  Taylor, 1981, {\it A Second Course in Stochastic Processes},
Academic Press, New York.

\bibitem{}
Kirkby, M. J., 1987, Modelling some influences of soil erosion, landslides and valley gradient
on drainage density and hollow development. {\sl Catena Supplement}, 10, 1-14.

%\bibitem{}
%Kuhnle, R. A., R. L. Bingner, G. R. Foster, and E. H. Grissinger, 1996)
%Effect of land use changes on sediment transport in Goodwin Creek, 
%{\it Water Resources Research}, 32(10), 3189-3196.

\bibitem{}
Lettenmaier, D. P., and E. F. Wood, 1992, Hydrologic forecasting, 
in {\it Handbook of Hydrology}, edited by D. R. Maidment, pp. 26.1-26.30, 
McGraw-Hill, Inc., New York.

\bibitem{}
Meyer, L.D., and W. H. Wischmeier, W.H., 1969, Mathematical simulation
of the process of soil erosion by water. {\it Transactions of the ASAE},
12, 754-758.

%\bibitem{}
%Mitas, L., Electronic structure by Quantum Monte Carlo: 
%atoms, molecules and solids, {\it Computer Physics Communications}, 97, 107-117, 1996.

\bibitem{}
     Mitas, L., and H. Mitasova, 1999, Spatial Interpolation. in:
     {\sl Geographical Information Systems: Principles, Techniques,
     Management and Applications}, edited by P.Longley, M.F. Goodchild, 
     D.J. Maguire, D.W.Rhind, John Wiley, 481-492.

\bibitem{}
     Mitas, L., and H. Mitasova, 1998, Distributed erosion modeling for
     effective erosion prevention. {\it Water Resources Research}, 34(3),
     505-516.

\bibitem{}
Mitasova, H., and L. Mitas, 1999a, Modeling soil detachment by RUSLE 3d using
GIS. \\
http://www2.gis.uiuc.edu:2280/modviz/erosion/usle.html

\bibitem{}
Mitasova, H., and L. Mitas, 1999b, Erosion/deposition modeling with USPED 
using GIS. \\
http://www2.gis.uiuc.edu:2280/modviz/erosion/usped.html

\bibitem{}
Mitasova, H., Mitas, L., Brown, W. M., and Johnston, D., 1999, 
{\it Terrain modeling and Soil
Erosion Simulations for Fort Hood and Fort Polk test areas}. Report for USA CERL.
University of Illinois, Urbana-Champaign, IL; \\
http://www2.gis.uiuc.edu:2280/modviz/reports/cerl99/rep99.html

\bibitem{} 
Mitasova, H., J. Hofierka, M. Zlocha, and L.R. Iverson, 1997,
     Modeling topographic potential for erosion and deposition using
     GIS. Reply to a comment.
   {\it  Int. Journal of Geographical Information Science}, 11(6), 611-618. 

\bibitem{}
Mitasova, H., J. Hofierka, M. Zlocha, and L.R. Iverson, 1996, Modeling 
    topographic potential for erosion and deposition using GIS. 
{\it Int. Journal of Geographical Information Science}, 10(5),
      629-641. 

\bibitem{}
Moore I.D., and Burch G.J., 1986, Modeling erosion and deposition: Topographic effects.
      {\it Transactions ASAE}, 29, 1624-1640. 

\bibitem{}
Moore, I. D., and G. R. Foster, 1990, Hydraulics and overland flow, 
in {\it Process Studies in Hillslope Hydrology}, 
edited by M. G. Anderson and T. P. Burt, John Wiley, 215-54.

\bibitem{}
Moore, I. D.,  A. K. Turner, J. P. Wilson, S. K. Jensen, and L. E. Band, 1993, 
GIS and land surface-subsurface process modeling, 
in: {\it Geographic Information Systems and Environmental Modeling}, 
edited by M. F Goodchild, L. T. Steyaert, and B. O. Parks, 
Oxford University Press, New York, 196-230.

\bibitem{}
National Research Council, 1999, {\it New Strategies for America's Watersheds}, 
Washington DC, National Academy Press.

\bibitem{}
NSDI, 2000, {\it National Spatial Data Infrastructure}. http://www.nsdi.org/

%\bibitem{}
%Nearing M.A., Norton L.D., Bulgakov D.A., Larionov G.A., West L.T., Dontsova K.M.,
%1997, Hydraulics and Erosion in eroding rills. Water Resources Research, 33, 865-876.

%Renard, G. K., G. R. Foster, G. A. Weesies, and J. P. Porter, RUSLE - Revised universal soil loss equation, {\it Journal of  Soil and Water Conservation}, 46, 30-33, 1991.

\bibitem{}
Roseboom, D. and Mollahan, R., 1999, Lake Pittsfield National Monitoring Project. 
Report for Illinosi State Water Survey and Illinois Environmental Protection
Agency, Peoria, Il.  

\bibitem{} 
Rouhi A., and J. Wright, 1995, Spectral implementation of a new operator splitting
method for solving partial differential equations, {\it Computers in Physics}, 9(5), 
554-563.

\bibitem{}
Saghafian, B., 1996, Implementation of a Distributed Hydrologic Model within GRASS, 
in {\it GIS and Environmental Modeling: Progress and Research Issues}, 
edited by M. F Goodchild, L. T. Steyaert, and B. O. Parks, GIS World, Inc., pp. 205-208.

\bibitem{}
SCS, 1988, {\it Manual for design of conservation measures}. Soil Conservation Service.

%\bibitem{}
%Srinivasan, R. and  B. A. Engel, 1991) A knowledge based approach
%to extract input data from GIS, ASAE Paper No. 91-7045, American Society
%of Agricultural Engineers, St.Joseph, Missouri, 1-8.

\bibitem{}
Srinivasan, R., and J. G. Arnold, 1994, Integration of a basin scale water quality model with GIS, 
{\it Water Resources Bulletin}, 30(3), 453-462.

\bibitem{} Stakgold, I., 1979, {\it Green's Functions and Boundary Value Problems},
John Wiley, New York.

\bibitem{}
Trimble, S.W., 1999, Decreased rates of alluvial sediment storage in
the Coon Creek basin, Wisconsin. {\it Science} 285(8), 1244-1246. 

\bibitem{}
Vieux, B. E., N. S. Farajalla, and N. Gaur, 1996,
 Integrated GIS and distributed storm 
water runoff modeling, in: {\it GIS and Environmental Modeling: 
Progress and Research Issues}, edited by M. F. Goodchild, L. T. Steyaert, 
and B. O. Parks, GIS World, Inc., pp. 199-205.

%\bibitem{}
%Warren, S. D., V. E. Diersing, P. J. Thompson, and  W. D. Goran,  An erosion-based land classification system for military installations, {\it Environmental Management}, 13, 251-257, 1989.


\bibitem{}
Willgoose, G. R., and Y. Gyasi-Agyei, 1995, New technology in hydrology and erosion 
assessment for mine rehabilitations, {\it Proceedings of the APCOM XXV Conference}, 
Brisbane, pp. 555-562.

\bibitem{}
Willgoose, G. R., R. L. Bras, and I. Rodriguez-Iturbe, 1989,
{\it A physically based channel network and catchment
evolution model}, technical report no. 322, Ralph Parsons Lab., MIT, Cambridge,
Mass., USA.

\bibitem{}
Wilson, J.P. and M.S. Lorang, 1999, Spatial Models of Soil Erosion and GIS. 
In Wegener M. and A.S. Fotheringham (Eds.), Spatial Models and GIS: 
New Potential and New Models (London: Taylor and Francis), 83-108.

\bibitem{}
Zhang Yusheng, 1999, GIS, Erosion and Deposition Modelling, 
and Caesium Technique.
http://www.ex.ac.uk/~yszhang/welcome.htm

\bibitem{} 
%Wischmeier, W. H., and D. D. Smith,  Predicting  rainfall erosion losses, a guide to conservation planning, {\it Agriculture Handbook},  no. 537,  US Department of Agriculture, Washington D. C., 1978.

}
\end{chapthebibliography}

\end{document}



