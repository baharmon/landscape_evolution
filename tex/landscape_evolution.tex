% !TEX encoding = UTF-8 Unicode

%%
%% Copyright 2007, 2008, 2009 Elsevier Ltd
%%
%% This file is part of the 'Elsarticle Bundle'.
%% ---------------------------------------------
%%
%% It may be distributed under the conditions of the LaTeX Project Public
%% License, either version 1.2 of this license or (at your option) any
%% later version.  The latest version of this license is in
%%    http://www.latex-project.org/lppl.txt
%% and version 1.2 or later is part of all distributions of LaTeX
%% version 1999/12/01 or later.
%%
%% The list of all files belonging to the 'Elsarticle Bundle' is
%% given in the file `manifest.txt'.
%%
%% Template article for Elsevier's document class `elsarticle'
%% with numbered style bibliographic references
%% SP 2008/03/01
%%
%% $Id: elsarticle-template-num.tex 4 2009-10-24 08:22:58Z rishi $
%%
%% \documentclass[preprint,12pt,3p]{elsarticle}
%% \documentclass[preprint,review,12pt]{elsarticle}
%% \documentclass[final,1p,times]{elsarticle}
%% \documentclass[final,1p,times,twocolumn]{elsarticle}
%% \documentclass[final,3p,times]{elsarticle}
%% \documentclass[final,5p,times]{elsarticle}
%% \documentclass[final,5p,times,twocolumn]{elsarticle}

\documentclass[final,3p,times,twocolumn]{elsarticle}

% Packages
\usepackage{graphicx}
\usepackage[utf8]{inputenc}
\usepackage{textcomp,marvosym}
\usepackage{gensymb}
\usepackage{caption}
\usepackage{subcaption}
\usepackage{hyperref}
\usepackage[super]{nth}
\usepackage[inline]{enumitem}
\usepackage{moreenum}
\usepackage{tabulary}
\usepackage{tabu}
\usepackage{booktabs}
\usepackage{array}
\usepackage[super]{nth}
\usepackage{listings}
\usepackage{float}
\usepackage{upquote}
\usepackage{minted}
\usemintedstyle{bw}
\newcommand{\ra}[1]{\renewcommand{\arraystretch}{#1}}

% Special characters
\usepackage{gensymb}
\usepackage{amsmath,amssymb}
\usepackage{pifont}
\newcommand{\cmark}{\ding{51}}
\newcommand{\xmark}{\ding{55}}

\journal{Geomorphology}

\begin{document}

\begin{frontmatter}

\title{Dynamic Landscape Evolution}

\author[cga,la]{Brendan Alexander Harmon\corref{cor1}}
\cortext[cor1]{Corresponding author}

\ead{brendan.harmon@gmail.com}
\ead[url]{baharmon@github.io}

\author[cga,meas]{Helena Mitasova}
\ead{hmitaso@ncsu.edu}

\author[cga,meas]{Vaclav Petras}
\ead{vpetras@ncsu.edu}

\author[cga,meas]{Anna Petrasova}
\ead{akratoc@ncsu.edu }

\address[cga]{Center for Geospatial Analytics, North Carolina State University, Raleigh, North Carolina, United States of America}
\address[la]{Department of Landscape Architecture, North Carolina State University, Raleigh, North Carolina, United States of America}
\address[meas]{Department of Marine, Earth, and Atmospheric Sciences, North Carolina State University, Raleigh, North Carolina, United States of America}


\begin{abstract}
This is a fine-scale, short term, process-based landscape evolution model using simulated erosion and deposition to generate a time-series of digital elevation models and compute the net change in elevation. This model uses a path sampling method to solve water and sediment flow continuity equations and model mass flows over complex topographies based on topographic, land cover, soil, and rainfall parameters. This either steady state or dynamic model can simulate landscape evolution for a range of hydrologic soil erosion regimes. 
%The change in elevation is a function of time, net erosion-deposition, and sediment mass density.
\end{abstract}

\begin{keyword}
%% keywords here, in the form: keyword \sep keyword
landscape evolution \sep dynamic model
%% MSC codes here, in the form: \MSC code \sep code
%% or \MSC[2008] code \sep code (2000 is the default)
\end{keyword}

\end{frontmatter}

% -------------- TOC --------------
\tableofcontents
\vfil
\pagebreak

% -------------- BODY --------------
\section{Introduction}
%With process-based simulations we have procedurally modeled how a landscape could evolve as the flow of water erodes the landscape surface and shapes its terrain.
This process-based, spatially distributed, dynamic model uses a path sampling method to solve the water and sediment flow equations
\cite{mitasova2004}
and model mass flows over complex topographies based on topographic, land cover, soil, and rainfall parameters.
% At each time step %% explain dynamics
The modeled flow of sediment -- a function of the flow of water and soil detachment and transport parameters -- is then used to estimate the net erosion and deposition rates and the associated short-term evolution of the topography. 

This highly adaptable model can simulate landscape evolution 
for different soil erosion regimes
across a range of spatiotemporal scales
using either
the simulated water erosion (SIMWE) model, 
the unit stream power erosion deposition (USPED) model,
or the 3-dimensional revised universal soil loss equation (RUSLE 3D) model.  

% Implementation 
This model has been implemented as an add-on module 
for a free, open-source geographic information system (GIS) -- GRASS GIS.
It supports multithreading and parallel processing
for the efficient computation of 
physics-based simulations for large, high resolution topographic datasets.



\subsection{Literature review}

Steady state versus dynamic flows \\

Spatial and temporal scales \\

Table of landscape evolution models \\

\subsection{Conceptual model}
% Diagram

%\section{Theory}




\section{Erosion-deposition model} \label{erdep_model}


\subsection{Shallow water flow}

We simulated shallow overland water flow controlled by spatially variable topography, soil, landcover, and rainfall parameters using the SIMWE model to solve the continuity and momentum equations for steady state water flow with a path sampling method. 
% implemented in GRASS GIS as the module r.sim.water.
%
Shallow water flow can be approximated by
the bivariate form of the St Venant equation:

\begin{equation}
\label{eq:water}
{\partial h({\bf r},t) \over \partial t} =
 i_e({\bf r},t) - \nabla \cdot {\bf q}({\bf r},t)
\end{equation}

\noindent
where:

\hspace*{1em} ${\bf r}(x,y)$ is the position [m]

\hspace*{1em} $t$ is the time [s]

\hspace*{1em} $h({\bf r},t)$ is the depth of overland flow [m]

\hspace*{1em} $i_e({\bf r},t)$ is the rainfall excess [m/s]\\
\hspace*{1em} (rainfall $-$ infiltration $-$ vegetation intercept) 

\hspace*{1em} ${\bf q}({\bf r},t)$ is the water flow per unit width [$\rm m^2/s$]. \\



%  diffusion wave approximation
By integrating a diffusion term $ \propto \nabla^2 [h^{5/3}({\bf r})]$ 
into
the solution of the continuity and momentum equations for steady state water flow
diffusive wave effects can be approximated
so that water can flow through depressions. 
%
\begin{equation}
\label{eq:difwater}
-{\varepsilon({\bf r})\over 2 }\nabla^2 [h^{5/3}({\bf r})]
+\nabla \cdot [ h({\bf r}){\bf v}({\bf r})] = i_e({\bf r})
\end{equation}

\noindent
 where:
 
 \noindent
 \hspace*{1em} $\varepsilon({\bf r})$ is a spatially variable diffusion coefficient.

This equation is solved using a Green's function Monte Carlo path sampling method \cite{mitasova2004}.

%Steady state water flow equation with a 2D diffusive wave approximation\ldots

%continuity and momentum equations for a steady water flow with a diffusive wave approximation


\subsection{Erosion-deposition}
%\subsection{Sediment flow}

Steady state sediment flow equation with diffusion\ldots
\begin{equation}\label{eq:sediment} 
...
\end{equation}


\subsection{Landscape evolution}

% change in elevation (m) = change in time (s) * net erosion-deposition (kg/m^2s) / sediment mass density (kg/m^3)
\begin{equation}
\label{eq:evolution} 
{\Delta z(x,y,t) = \Delta t \cdot d_s(x,y,t) \cdot \rho_s^{-1} }
\end{equation}

\noindent
where: 

\noindent
$\Delta z =$ change in elevation $(m)$ \\
$d_s =$ net erosion-deposition $(kg ~ m^{-2} s^{-1})$ \\
$\rho_s =$ sediment mass density $(kg ~m^{-3})$ \\
\vspace{1em}

\ldots
\cite{mitasova2013}

\subsection{Gravitational diffusion}

% change in elevation (m) = elevation (m) - (change in time (s) / sediment mass density (kg/m^3) * gravitational diffusion coefficient (m^2/s) * divergence (m^-1))

\begin{equation}
\label{eq:grav_diffusion} 
{\Delta z(x,y,t) = \Delta t \cdot \rho_s^{-1} \cdot \varepsilon_g \cdot div(x,y,t)}
\end{equation}

\noindent
where: 

\noindent
$\Delta z =$ change in elevation $(m)$ \\
$\rho_s =$ sediment mass density $(kg ~m^{-3})$ \\
$\varepsilon_g =$ gravitational diffusion coefficient $(m^{-2} s^{-1})$ \\
$div =$ divergence $(m^{-1})$ \\
\vspace{1em}

\ldots
\cite{thaxton2004}


\section{Detachment limited model} \label{flux_model}
\subsection{Shallow water flow}
\subsection{Sediment flow}

% Detachment limited landscape evolution

\begin{equation}
\label{eq:flux_evolution} 
{\Delta z(x,y,t)} = \Delta t \cdot q_s(x,y,t) \cdot \varrho(r)^{-1}
\end{equation}

\noindent
where: 

\noindent
$\Delta z =$ change in elevation $(m)$ \\
$q_s =$ sediment flux $(kg \cdot m^{-1} s^{-1})$ \\
$\varrho =$ mass of water carried sediment per unit area $(kg \cdot m^{-2})$ \\
\vspace{1em}

\ldots
\cite{mitasova2013}

\subsection{Landscape evolution}

% change in elevation (m) = change in time (s) * sediment flux (kg/ms) / mass of sediment per unit area (kg/m^2)

\begin{equation}
\label{eq:evolution} 
{\Delta z(x,y,t) = \Delta t \cdot q_s(x,y,t) \cdot \varrho_s^{-1} }
\end{equation}

\noindent
where: 

\noindent
$\Delta z =$ change in elevation $(m)$ \\
$q_s =$ sediment flux $(kg \cdot m^{-1} s^{-1})$ \\
$\varrho =$ mass of water carried sediment per unit area $(kg \cdot m^{-2})$ \\
\vspace{1em}

\ldots
\cite{mitasova2013}


\subsection{Gravitational diffusion}

\section{Transport limited model} \label{transport_model}


\section{Unit stream power erosion deposition model} \label{usped_model}

\subsection{Unit stream power erosion deposition} % transport limited
\subsection{Landscape evolution}
\subsection{Gravitational diffusion}

\clearpage
% -------------------------------- RUSLE --------------------------------
\section{Revised universal soil loss equation 3D model}
\label{rusle_model} % detachment limited

% https://ncsu-geoforall-lab.github.io/erosion-modeling-tutorial/erdep_theory.html  % ---- RESOURCES ------
% http://www4.ncsu.edu/~hmitaso/gmslab/reports/CerlErosionTutorial/denix/denixstart.html % ---- RESOURCES ------

\subsection{Revised universal soil loss equation} % 3D

\paragraph{Event-based r-factor derivation} % (MJ mm ha^-1 hr^-1)

%`the R-factor is the product of kinetic energy of a rainfall event
%and its maximum 30-minute intensity'
%\cite{panagos_2015}

% r-factor \cite{Renard1997}

% rain energy
\cite{Brown1987}
%
\begin{equation}
\label{eq:rain_energy}
{e_r = 0.29 \cdot (1.-0.72 \cdot exp(-0.05 \cdot i_r))}
\end{equation}

\noindent
where: 

\noindent
$e_r =$ unit rain energy $(MJ \cdot ha^{-1} \cdot mm{^-1})$\\
$i_r =$ rainfall intensity $(mm \cdot h^{-1})$\\
\vspace{1em}


% rainfall volume
\begin{equation}
\label{eq:rain_volume}
{v_r = i_r \cdot t_r} %*(1 hr / 60 min))
\end{equation}

\noindent
where: 

\noindent
$v_r =$ rainfall volume $(mm)$\\
$i_r =$ rainfall intensity $(mm \cdot h^{-1})$\\
$t_r =$ time interval $(h^{-1})$\\
\vspace{1em}


% event erosivity index (MJ mm ha^-1 hr^-1)
\begin{equation}
\label{eq:erosivity_index}
{r = e_r \cdot v_r \cdot i_r}
\end{equation}

\noindent
where: 

\noindent
$r =$ erosivity index $(MJ \cdot mm \cdot ha^{-1} \cdot hr^{-1} \cdot  s{^-1})$\\
$e_r =$ unit rain energy $(MJ \cdot ha^{-1} \cdot mm{^-1})$\\
$v_r = $ rainfall volume $(mm)$\\
$i_r = $ rainfall intensity $(mm \cdot h^{-1})$\\
\vspace{1em}
% multiply by rainfall interval in seconds (MJ mm ha^-1 hr^-1 s^-1)


\paragraph{3D topographic factor}

%The modified 3D topographic factor LS3D, representing topographic potential for erosion at a point on the hill-slope, is a function of the upslope contributing area per unit width and the slope angle

\begin{equation}
\label{eq:ls_factor}
{ls(x,y) = (m+1.0) \cdot (a(x,y) \cdot a_0^{-1})^{m} \cdot (sin(\beta) / \beta_0)^{n}}
\end{equation}

\noindent
where: 

%U is the upslope area per unit width (measure of water flow) in meters (m^2/m), 
%β is the slope angle in degree, 
%22.1 is the length of the standard USLE plot in meters, 
%0.09 = 9% = 5.15◦ is the slope of the standard USLE plot. 
%The values of exponents range for m = 0.2 − 0.6 and n = 1.0 − 1.3, where the lower values are used for prevailing sheet flow and higher values for prevailing rill flow.

\noindent
$ls =$ a dimensionless topographic (length-slope) factor\\
$a =$ water flow accumulation $(m)$\\
$a_0 =$ length of the standard USLE plot $(22.1 m)$\\
$\beta =$ slope angle $(\degree)$\\
$m =$ empirical coefficient\\
$n =$ empirical coefficient\\
$\beta_0 =$ slope of the standard USLE plot $(0.09 \degree)$\\
\vspace{1em}
%
\cite{mitasova_1996} %http://www4.ncsu.edu/~hmitaso/gmslab/papers/erijgis.html

%A simpler, continuous form of the equation for computation of the LS factor at a point r=(x,y) on a hillslope, (Mitasova et. al. 1996) is
%
%LS(r)  =  (m+1)  [ A(r) / a0 ]m  [ sin b(r) / b0 ]n
%
%where A[m]  is upslope contributing area per unit contour width, b [deg] is the slope, m and n are parameters, and a0  = 22.1m = 72.6ft  is the length and b0 = 0.09 = 9% = 5.16deg is the slope of the standard USLE plot. The impact of replacing the slope length by upslope area on the detachment pattern is illustrated by the following figures, which show that the upslope area-based factor better reflects the impact of concentrated flow on increased erosion:



\paragraph{Revised universal soil loss equation}
%
\begin{equation}
\label{eq:rusle}
{e = r \cdot k \cdot ls \cdot c \cdot p}
\end{equation}

\noindent
where: 

\noindent % CHECK UNITS
$e =$ soil loss $(kg \cdot m^{-2} \cdot min^{-1})$\\
$r =$ erosivity factor $(MJ \cdot mm \cdot ha^{-1} \cdot hr^{-1} \cdot  s{^-1})$\\
$k =$ soil erodibility factor $(ton \cdot ha \cdot hr \cdot ha^{-1} \cdot MJ^{-1} \cdot mm^{-1})$\\
$ls =$ dimensionless topographic (length-slope) factor\\
$c =$ dimensionless land cover factor\\
$p =$ dimensionless prevention measures factor\\
\vspace{1em}





\subsection{Landscape evolution}
\subsection{Gravitational diffusion}

\clearpage
% -------------------------------- IMPLEMENTATION --------------------------------
\section{Implementation}

% EDIT
%\begin{enumerate}
%\item Function for sediment flux based landscape evolution
%\item Function for erosion-deposition based landscape evolution
%\item Function for dynamic modeling based on constant parameters
%\item Function for dynamic modeling based on list of rainfall observations
%\item Registration in temporal framework
%\item Handling of edge effects from moving window computations
%\end{enumerate}

%This set of python scripts
The GRASS GIS add-on module written in Python
is available on Github at \url{https://github.com/baharmon/landscape_evolution} released under the GNU General Public License version 2. 
%These scripts are meant to be run inside of GRASS GIS using the GRASS Python Scripting Library.
GRASS GIS is an open source project released under the GNU General Public License version 2. GRASS GIS is available at \url{https://grass.osgeo.org/}. 
% Geographic Resource Analysis Support System (GRASS)


\clearpage
% -------------------------------- CASE STUDY--------------------------------
\section{Case study}
\subsection{Fort Bragg}
\subsection{Patterson Branch Creek}
%\subsection{Comparison of r.evolution, r.land.evol, r.terradyn}
\subsection{Benchmarks}
% r.evolution, r.land.evol, r.terradyn

%\section{Results}

%\begin{figure}
%\centering
%%   
%\begin{subfigure}[b]{0.3\textwidth}
%\includegraphics[width=\textwidth]{images/lrwoods_elevation.png}
%%[trim={0 0 0 1cm},clip,width=\textwidth]
%\label{fig_1_1}
%\textbf{a} \\
%\end{subfigure}
%%
%~ %add desired spacing between images, e. g. ~, \quad, \qquad, \hfill etc.
%%
%\begin{subfigure}[b]{0.3\textwidth}
%\includegraphics[width=\textwidth]{images/lrwoods_dynamics_flux_5m_30m.png}
%\label{fig_1_2}
%\textbf{b} \\
%\end{subfigure}
%%
%~ %add desired spacing between images, e. g. ~, \quad, \qquad, \hfill etc.
%%
%\begin{subfigure}[b]{0.3\textwidth}
%\includegraphics[width=\textwidth]{images/lrwoods_dynamics_erdep_5m_30m_flux.png} % REPLACE IMAGE
%\label{fig_1_3}
%\textbf{c} \\
%\end{subfigure}
%%
%\caption{{\bf Sediment flux based gully evolution.}
%\textbf{a)}
%A bare earth digital elevation model of gully in Lake Raleigh Woods, North Carolina derived from lidar data.
%\textbf{b)}
%The simulated evolution of the gully based on a detachment limited soil erosion regime.
%The landscape evolution model was run as a dynamic simulation with 155 mm/hr rainfall intensity for 5 minutes intervals over a 30 min period.
%This run of model carved deep pits along the center of the channel.
%\textbf{c)}
%Simulated sediment flux. 
%}
%\label{fig_1}
%\end{figure}
%
%\begin{figure}
%\centering
%%   
%\begin{subfigure}[b]{0.3\textwidth}
%\includegraphics[width=\textwidth]{images/lrwoods_elevation.png}
%%[trim={0 0 0 1cm},clip,width=\textwidth]
%\label{fig_2_1}
%\textbf{a} \\
%\end{subfigure}
%%
%~ %add desired spacing between images, e. g. ~, \quad, \qquad, \hfill etc.
%%
%\begin{subfigure}[b]{0.3\textwidth}
%\includegraphics[width=\textwidth]{images/lrwoods_dynamics_erdep_5m_30m.png}
%\label{fig_2_2}
%\textbf{b} \\
%\end{subfigure}
%%
%~ %add desired spacing between images, e. g. ~, \quad, \qquad, \hfill etc.
%%
%\begin{subfigure}[b]{0.3\textwidth}
%\includegraphics[width=\textwidth]{images/lrwoods_dynamics_erdep_5m_30m_erdep.png} % REPLACE IMAGE
%\label{fig_2_3}
%\textbf{c} \\
%\end{subfigure}
%%
%\caption{{\bf Erosion - deposition based gully evolution.}
%\textbf{a)}
%A bare earth digital elevation model of gully in Lake Raleigh Woods, North Carolina derived from lidar data.
%\textbf{b)}
%The simulated evolution of the gully based on a transport capacity limited  soil erosion regime.
%The landscape evolution model was run as a dynamic simulation with 155 mm/hr rainfall intensity for 5 minutes intervals over a 30 min period.
%This run of model carved a deeper channel, accumulated deposited sediment along the centerline of the channel, and accumulated deposited sediments along the banks of the channel.
%\textbf{c)}
%Simulated erosion-deposition. 
%}
%\label{fig_1}
%\end{figure}
%
%\begin{figure}
%\centering
%%   
%\begin{subfigure}[b]{0.4\textwidth}
%\includegraphics[width=\textwidth]{images/dem.png}
%%[trim={0 0 0 1cm},clip,width=\textwidth]
%\label{fig_2_1}
%\textbf{a} \\
%\end{subfigure}
%%
%~ %add desired spacing between images, e. g. ~, \quad, \qquad, \hfill etc.
%%
%\begin{subfigure}[b]{0.4\textwidth}
%\includegraphics[width=\textwidth]{images/evolved_dem.png}
%\label{fig_2_2}
%\textbf{b} \\
%\end{subfigure}
%%
%\caption{{\bf Sediment flux based gully evolution.}
%\textbf{a)}
%A gully in Lake Raleigh Woods, North Carolina.
%\textbf{b)}
%The simulated evolution of the gully based on a detachment limited soil erosion regime. 
%The landscape evolution model was run as a steady state simulation with 155 mm/hr rainfall intensity for 10 minutes to model a 10-year storm event. 
%This run of the model carved a deep incision along the centerline of the channel.
%}
%\label{fig_2}
%\end{figure}
%
%\begin{figure}
%\centering
%%   
%\begin{subfigure}[b]{0.4\textwidth}
%\includegraphics[width=\textwidth]{images/elevation_render.png}
%%[trim={0 0 0 1cm},clip,width=\textwidth]
%\label{fig_3_1}
%\textbf{a} \\
%\end{subfigure}
%%
%~ %add desired spacing between images, e. g. ~, \quad, \qquad, \hfill etc.
%%
%\begin{subfigure}[b]{0.4\textwidth}
%\includegraphics[width=\textwidth]{images/evolved_elevation_render.png}
%\label{fig_3_2}
%\textbf{b} \\
%\end{subfigure}
%%
%\caption{{\bf Erosion-deposition based gully evolution.}
%\textbf{a)}
%A gully in Lake Raleigh Woods, North Carolina.
%\textbf{b)}
%The simulated evolution of the gully based on a transport capacity limited  soil erosion regime.
%The landscape evolution model was run as a dynamic simulation with 155 mm/hr rainfall intensity for 5 minutes intervals over a 30 min period.
%This run of model carved a deeper channel, accumulated deposited sediment along the centerline of the channel, and accumulated deposited sediments along the banks of the channel.
%}
%\label{fig_3}
%\end{figure}

\clearpage
% -------------------------------- TANGIBLE --------------------------------
\section{Tangible landscape evolution}

Tangible Landscape -- a tangible user interface tightly integrated with a geographic information system for intuitively sketching in 3D \cite{petrasova2015}. Conceptually, Tangible Landscape couples a physical model with a digital model in a real-time feedback cycle of 3D scanning, geospatial modeling and simulation, and projection in order to physically manifest digital data as tangible bits. With tangible bits users can directly, physically feel and manipulate data with their bodies -- naturally, intuitively understanding space, form, and process. Tangible Landscape is available on Github at \url{https://github.com/ncsu-osgeorel/grass-tangible-landscape}.

%\paragraph{Testing}
We coupled Tangible Landscape with the landscape evolution model to test the model and experiment with strategies for restoration. 
We used Tangible Landscape to computationally steer the landscape evolution model and interactively explore the relationship between overland flow patterns and changes in topography. By manually changing the physical model of the landscape 
we change the topography used by the model.

\begin{figure*}
\centering
\includegraphics[width=\textwidth]{images/cnc_sand.jpg}
\caption{{\bf Rapid prototyping.}
3-axis CNC fabrication of the evolved landscape in polymer-enriched sand using a plunge cut.}
\label{fig:cnc_sand}
\end{figure*}

% -------------------------------- DISCUSSION --------------------------------
\section{Discussion}

\subsection{Future work}
\begin{enumerate}
\item Test the model on historical data
\item Test the model with UAS Sfm time-series
%\item Empirically calibrate the parameters
\item Implement as a Tangible Landscape analysis
\item Live, in-situ fabrication in polymer-enriched sand with a robotic arm
\end{enumerate}

\section{Conclusion}

% -------------------------------- APPENDIX --------------------------------
\appendix

\section{Supporting information}

\subsection{Code}\label{code}
{\bf Github repository}

\subsection{Data}\label{data}
{\bf GRASS GIS Mapset}

\subsection{3D models}\label{3d_models}
{\bf \ldots}

\subsection{Tangible Landscape}\label{tangible_landscape}
{\bf \ldots}

% -------------------------------- BIBLIOGRAPHY --------------------------------

% \bibliographystyle{elsarticle-num}
 \bibliographystyle{elsarticle-harv}
% \bibliographystyle{elsarticle-num-names}
% \bibliographystyle{model1a-num-names}
% \bibliographystyle{model1b-num-names}
% \bibliographystyle{model1c-num-names}
% \bibliographystyle{model1-num-names}
% \bibliographystyle{model2-names}
% \bibliographystyle{model3a-num-names}
% \bibliographystyle{model3-num-names}
% \bibliographystyle{model4-names}
% \bibliographystyle{model5-names}
% \bibliographystyle{model6-num-names}

\bibliography{landscape_evolution.bib}


\end{document}

